\documentclass[10pt, a4paper]{article}
\usepackage[margin=0.8in]{geometry}
\usepackage{enumitem}
\usepackage{titlesec}
\usepackage{parskip}

% Compact title
\titleformat{\section}{\large\bfseries}{\thesection}{1em}{}
\titlespacing*{\section}{0pt}{1.5ex plus 0.5ex minus .2ex}{0.5ex plus .2ex}
\setlength{\parindent}{0pt}
\setlength{\parskip}{0.5em}

\begin{document}
\begin{titlepage}
    \centering
    \vspace*{\fill}
    {\Huge \textbf{Technologies of Democratic Societies}\par}
    \vspace*{\fill}
\end{titlepage}

\newpage
\section*{Week 1 \& 2: The Vision \& The Experiment}

\textbf{The Big Picture.}
This course asks a simple question: \textit{Does technology help democracy or hurt it?}
We don't just ask "how do we build it?" but "what should we build?" We look at the gap between the \textbf{dream} (everyone has a voice) and the \textbf{reality} (spam, fake news, and surveillance).

\section*{Part 1: The Dream (Licklider's Vision, 1968)}
Long before the internet, J.C.R. Licklider dreamed of a future where computers would help people think and work together better.

\textbf{Communication is "Modeling," not just Sending.}
Licklider argued that real communication isn't just sending a file to someone. It's about properly understanding what is in the other person's head (their "mental model"). He believed computers would help us show these models to each other, making teamwork faster and smarter than face-to-face meetings.

\textbf{The "OLIVER" (Digital Assistant).}
He predicted we would all have a digital assistant (he called it an OLIVER) that knows what we like and helps us filter information. Today, these are the algorithms and AI bots we interact with.

\textbf{Communities based on Interest, not Location.}
He predicted that people would form groups based on what they \textit{love}, not just where they \textit{live}. This would make life happier and more productive.

\textbf{The Warning: The Digital Divide.}
He also worried: If only rich people have computers, will this new power be a privilege for a few, or a right for everyone?

\section*{Part 2: The Experiment (Usenet, 1979)}
Usenet was the first real attempt to build this dream. It was the "grandfather" of Reddit and Twitter—a global network where anyone could post a message for the world to see.

\textbf{How it worked (Simple Version):}
\begin{itemize}[noitemsep, topsep=0pt, leftmargin=*]
    \item \textbf{UUCP (Copying Files):} It started with a simple trick. One computer would call another over a phone line and "copy" a file to it.
    \item \textbf{Gossip Protocol:} That computer would then call another, and another. Like a rumor spreading in a school, eventually, every computer had the message.
    \item \textbf{Newsgroups:} Instead of emailing one person ("To: Bob"), you posted to a topic ("To: Unix-Lovers").
\end{itemize}

\textbf{The Philosophy: No Bosses, No Censorship.}
Usenet was \textbf{Decentralized}. No single company owned it. It was run by regular people (admins) connecting their computers.
It was designed to be \textbf{Censorship Resistant}. If one computer blocked a message, the "gossip" would just find another path around it. This meant total Freedom of Speech.

\textbf{Why it Failed: The Spam Explosion.}
Because the system was designed to never block anyone, it couldn't block bad actors.
\begin{itemize}[noitemsep, topsep=0pt, leftmargin=*]
    \item \textbf{Limitless Spam:} In 1994, lawyers Canter \& Siegel posted an ad for "Green Cards" to every single group. This proved you could use Usenet for free advertising.
    \item \textbf{Signal-to-Noise Collapse:} Soon, Usenet was full of junk. It became too hard to find the real conversations. People left for places like Reddit where moderators could delete the junk.
\end{itemize}

\section*{Part 3: The Scorecard (Dahl's Criteria)}
How do we judge if a technology like Usenet is actually "democratic"? We use Robert Dahl's 5 rules for a democracy.

\textbf{The Base Rule: We are all Competent.}
Democracy assumes that every adult is smart enough to decide what is best for themselves.

\textbf{1. Effective Participation (Can you be heard?)}
\textit{Rule:} Everyone must have a real chance to speak and be heard \textit{before} a decision is made.
\textit{Usenet's Grade:} \textbf{Fail.} Technically everyone could speak, but because of all the spam/noise, nobody could actually be heard.

\textbf{2. Voting Equality (One Person, One Vote).}
\textit{Rule:} When it's time to decide, my vote and your vote must count exactly the same.
\textit{Usenet's Grade:} \textbf{Fail.} Because there were no ID checks, one person could make 100 fake accounts ("Sock Puppets") and vote 100 times.

\textbf{3. Enlightened Understanding (Do you know the truth?)}
\textit{Rule:} You need access to the truth and different facts to make a smart choice.
\textit{Usenet's Grade:} \textbf{Mixed.} It promised access to all human knowledge, but without filters, it was easy to get lost in lies or junk data.

\textbf{4. Control of the Agenda (Who decides what we discuss?)}
\textit{Rule:} The people (not a king or a CEO) must decide what topics are important.
\textit{Usenet's Grade:} \textbf{Good start, poor finish.} At first, anyone could start a topic. Later, spammers hijacked the attention, effectively controlling what everyone saw.

\textbf{5. Inclusiveness (Is everyone invited?)}
\textit{Rule:} Every adult subject to the rules must be allowed to join.
\textit{Usenet's Grade:} \textbf{Pass.} It was famously open. If you could get to a computer, you were in.

\hrulefill

\textbf{Bottom Line.}
Usenet proves that \textbf{Uncensored Speech} (freedom to say anything) is not the same as \textbf{Democracy} (freedom to decide together). Without rules to stop abuse (like spam or fake voters), the democratic process breaks down.


\newpage
\section*{Week 3: The Metric of Influence}

\textbf{The Big Picture.}
This week asks: \textit{What is influence, and how do we measure it?} We look at three ways influence is measured today: \textbf{Social} (Twitter), \textbf{Algorithmic} (Google), and \textbf{Economic} (Wealth). A key claim: \textbf{Democracy itself is an influence metric.} Voting is how we measure who convinced whom. Campaigning is the attempt to \textit{create} influence; the vote is the \textit{measurement} of its success.

\textbf{The Danger:} These systems follow \textbf{Power Laws} (the "Rich get Richer"). A tiny number of people get almost all the attention, leaving everyone else invisible.

\section*{Part 1: Social Influence (The "Million Follower" Fallacy)}
We assume "more followers = more influence." A study of 54 million Twitter users proves this is a \textbf{Fallacy}.

\textbf{Three Measures of Influence:}
\begin{itemize}[noitemsep, topsep=0pt, leftmargin=*]
    \item \textbf{Indegree (Followers):} Measures \textbf{Popularity}. People are watching, but not necessarily listening. (Top users: CNN, Obama, Britney Spears)
    \item \textbf{Retweets:} Measures \textbf{Content Value}. People share what you say. (Top users: Mashable, Guy Kawasaki, NYTimes)
    \item \textbf{Mentions:} Measures \textbf{Name Value/Engagement}. People talk \textit{about} you. (Top users: Celebrities like Ashton Kutcher)
\end{itemize}
\textit{Key Finding:} The top 20 lists for each metric had almost \textbf{no overlap} (only 2 users in common). Popularity $\neq$ Influence.

\textbf{Two Theories of Influence:}
\begin{itemize}[noitemsep, topsep=0pt, leftmargin=*]
    \item \textbf{Traditional ("Influentials" Theory):} A small elite of "opinion leaders" drives trends. Target them, and the message spreads.
    \item \textbf{Modern ("Accidental Influentials" Theory):} Trends depend on how \textit{ready} society is to adopt them, not on who starts them. Anyone can spark a trend if the timing is right.
\end{itemize}
\textit{The Data's Verdict:} The traditional view is mostly right. Influence is \textbf{not gained by accident, but through concerted effort}—consistently posting valuable content on a single topic. The top influentials dominated across many different topics.

\textbf{The Streisand Effect.} Trying to censor something can backfire. When Barbra Streisand tried to stamp out negative content about herself, it only drew more attention to it. Retweets can spread ideas you hate just as easily as ideas you love.

\section*{Part 2: Algorithmic Influence (PageRank \& Google)}
Before Google, search engines like AltaVista just counted keywords. Results were random and useless. Google fixed this by treating \textbf{Links as Votes}.

\textbf{How PageRank Works:}
\begin{itemize}[noitemsep, topsep=0pt, leftmargin=*]
    \item \textbf{Recursive Importance:} A page is important if \textit{important pages} link to it. One link from the New York Times is worth more than 1,000 links from random blogs.
    \item \textbf{The "Rank Sink" Problem:} A group of pages linking only to each other would hoard all the power like dictators.
    \item \textbf{The Solution (Random Surfer / UBI):} The algorithm imagines a web surfer who clicks randomly but occasionally "gets bored" and jumps to a completely random page. This gives every page a tiny baseline chance—a "Universal Basic Income" of attention—preventing total monopoly.
\end{itemize}

\textbf{Personalized PageRank \& Filter Bubbles.}
The "random jump" destination (the E vector) can be customized. If E is uniform, it's a democratic view of the web. If E is your bookmarks, it's a \textit{personalized} view. This is powerful but risks creating \textbf{Filter Bubbles} where you only see what you already agree with.

\textbf{Manipulation: SEO \& Google Bombing.}
PageRank is \textit{harder} to game than simple keyword counting, but not impossible.
\begin{itemize}[noitemsep, topsep=0pt, leftmargin=*]
    \item \textbf{SEO (Search Engine Optimization):} The art of gaming the algorithm. Edit Wikipedia, build link farms, etc.
    \item \textbf{Google Bombing:} A group coordinates to link a specific word to a specific page. Example: Activists made a political candidate's name the top search result for an offensive made-up word.
\end{itemize}

\textbf{The Pornography Paradox.}
PageRank revealed something interesting: Pornographic sites had very high \textit{usage} (from web traffic data) but very low \textit{PageRank}. Why? People consume them privately but don't link to them publicly. PageRank measures what people \textit{endorse}, not just what they \textit{consume}.

\section*{Part 3: Economic Influence (Piketty's Inequality)}
Economist Thomas Piketty analyzed 100+ years of tax data and found a \textbf{U-shaped Curve} of inequality.

\textbf{The Core Finding:}
\begin{itemize}[noitemsep, topsep=0pt, leftmargin=*]
    \item \textbf{The Anomaly:} The relative equality of the mid-20th century (1940s-1970s) was a historical accident caused by wars and policy, not the natural state of capitalism.
    \item \textbf{The Return to Inequality:} Since 1980, wealth has concentrated sharply at the top (Top 1\% now own $>$33\% of U.S. wealth). The U.S. is now \textit{more unequal than Europe}—a complete reversal from 100 years ago.
    \item \textbf{The Formula ($r > g$):} When the \textbf{return on capital (r)} (stocks, real estate) is greater than \textbf{economic growth (g)}, wealth concentrates. Those who own things get richer faster than those who work.
\end{itemize}
\textit{Why it Matters for Democracy:} In a world where money buys influence (ads, campaigns, lobbying), extreme inequality breaks the promise of "One Person, One Vote."

\section*{Part 4: The Scorecard (Dahl's Criteria)}
Does the modern "Influence Economy" help or hurt democracy?

\textbf{1. Effective Participation (Can you be heard?)}
\textit{Grade:} \textbf{Fail.} Power Laws mean the top 1\% of accounts get almost all the attention. If you are new, it's nearly impossible to break through.

\textbf{2. Voting Equality (One Person, One Vote).}
\textit{Grade:} \textbf{Fail.} PageRank explicitly gives more voting power to "important" pages. Wealth inequality gives more power to the rich in real elections.

\textbf{3. Enlightened Understanding (Do you know the truth?)}
\textit{Grade:} \textbf{Pass (with caveats).} Google made finding facts easier than ever. But Personalized PageRank and algorithms can trap you in Filter Bubbles.

\textbf{4. Control of the Agenda (Who decides what we discuss?)}
\textit{Grade:} \textbf{Mixed.} Algorithms (and those who game them via SEO/Google Bombing) increasingly decide what news you see. The agenda is set by code, not citizens.

\hrulefill

\textbf{Bottom Line.}
Technology has given us powerful tools to measure influence (Twitter metrics, PageRank, wealth data). But these tools consistently show \textbf{Power Laws}—a few get almost everything. A search engine is not a democracy. Influence is not equally distributed. And the gap between who \textit{speaks} and who is \textit{heard} remains the central challenge for democratic technology.


\newpage
\section*{Week 4: Trust and Online Reputation}

\textbf{The Big Picture.}
Trust is the currency of society. In a large democracy, we cannot personally know everyone we interact with, so we rely on \textbf{Proxy Systems} to measure reputation. We trust \textbf{Peer Review} for scientific truth and \textbf{Search Engines} for information relevance. This week reveals that these are not neutral measurements; they are \textbf{mechanisms} that can be designed, manipulated, and broken.

\section*{Part 1: Scholarly Peer Review (The Original Trust Algorithm)}
Scientific truth isn't determined by a vote, but by \textbf{Peer Review}. Independent experts blind-test ideas before they are published.

\textbf{The Hierarchy of Trust.}
\begin{itemize}[noitemsep, topsep=0pt, leftmargin=*]
    \item \textbf{Pre-prints (e.g., arXiv):} Fast, but unverified. Low trust.
    \item \textbf{Conferences:} Peer-reviewed, but faster cycle (common in CS). Medium-High trust.
    \item \textbf{Journals:} Slow, rigorous review. The Gold Standard. High trust.
\end{itemize}

\textbf{Blinding in Peer Review.}
To reduce bias, different blinding approaches are used:
\begin{itemize}[noitemsep, topsep=0pt, leftmargin=*]
    \item \textbf{Single-blind:} Reviewers know authors, but authors don't know reviewers.
    \item \textbf{Double-blind:} Neither knows the other. Most rigorous for minimizing bias.
\end{itemize}

\textbf{The Metric: h-index.}
We measure a scientist's value by their \textbf{h-index}: A researcher has an index of $h$ if they have published $h$ papers that have each been cited at least $h$ times. It balances \textit{quantity} (papers) and \textit{quality} (citations).

\textbf{The Hack: Gaming Science.}
A major study of over 12,000 academics reveals how widespread the problem is:
\begin{itemize}[noitemsep, topsep=0pt, leftmargin=*]
    \item \textbf{Honorary Authorship:} \textbf{35.5\%} of researchers admitted to adding authors who contributed nothing. The main reason? Adding a ``big name'' to boost credibility.
    \item \textbf{Coercive Citations:} \textbf{14.1\%} reported being forced by editors to add irrelevant citations to the editor's own journal.
    \item \textbf{Padded Citations:} \textbf{$>$40\%} said they would add superfluous citations if they thought the journal expected it.
    \item \textbf{Citation Rings:} Groups of authors agree to cite each other (regardless of relevance) to artificially boost their h-indices.
    \item \textbf{Result:} Bad science can appear reputable, misleading the public and policymakers.
\end{itemize}

\section*{Part 2: Search Engine Manipulation (The Modern Gatekeeper)}
Search engines like Google determine what exists in our digital world.
\begin{itemize}[noitemsep, topsep=0pt, leftmargin=*]
    \item \textbf{The Power of Page 1:} 91.5\% of all clicks happen on the first page of results. If you aren't there, you are invisible.
    \item \textbf{Trust:} We assume the top result is the ``best'' or ``truest,'' but it's often just the best \textit{optimized}.
\end{itemize}

\textbf{Black Hat SEO (Search Engine Optimization).}
Manipulators use unethical tricks to fool the algorithm:
\begin{itemize}[noitemsep, topsep=0pt, leftmargin=*]
    \item \textbf{Cloaking:} Showing one version of a page to the Google Bot (keyword-rich) and a different version to human users (ads/spam).
    \item \textbf{Link Farms:} Creating thousands of fake websites with the sole purpose of linking to a target site to boost its PageRank.
    \item \textbf{Invisible Text:} Hiding keywords by using white text on a white background (or tiny fonts) to trick bots without impacting the user experience.
    \item \textbf{Google Bombing:} Collectively linking a phrase (e.g., ``miserable failure'') to a specific target (e.g., a politician) to hijack the search result.
    \item \textbf{Bowling:} A destructive attack. Instead of boosting your own site, you \textit{harm} a competitor by associating them with ``bad neighborhoods'' (link farms, spam sites). This triggers a penalty for your rival.
\end{itemize}

\textbf{The Search Engine Manipulation Effect (SEME).}
A series of experiments revealed something alarming: biased search results don't just change what we click—they change how we \textit{vote}.
\begin{itemize}[noitemsep, topsep=0pt, leftmargin=*]
    \item \textbf{The Finding:} Biased search rankings shifted the preferences of undecided voters by \textbf{20\% or more}. In some demographics, the shift was as high as \textbf{80\%}.
    \item \textbf{Real-World Test:} During India's 2014 national election, researchers showed undecided voters biased search results. Even with \textit{real} candidates they knew, the manipulation worked.
    \item \textbf{Awareness Doesn't Help:} Even when users \textit{knew} the results were biased, the effect often still worked. People trust that top results are ``better.''
    \item \textbf{The Threat:} This is undetectable ``digital gerrymandering.'' Since results are personalized, a regulator sees neutral results while a targeted voter sees biased ones.
\end{itemize}

\section*{Part 3: The Scorecard (Dahl's Criteria)}
\textbf{How do Trust Systems Measure Up?}

\textbf{1. Effective Participation (Can you be heard?)}
\textit{Grade:} \textbf{Fail.} Scientists without ``big names'' struggle to get published. Websites without SEO budgets are invisible. The playing field is not level.

\textbf{2. Voting Equality (One Person, One Vote).}
\textit{Grade:} \textbf{Fail.} Citation rings and link farms create ``fake votes.'' A researcher in a citation ring counts more than an honest one. A site with a link farm ranks higher than a legitimate competitor.

\textbf{3. Enlightened Understanding (Do you know the truth?)}
\textit{Grade:} \textbf{Fail.} When science and search are manipulated by \textbf{Citation Rings}, \textbf{Link Farms}, or \textbf{SEME}, citizens make decisions based on false or biased information. Worse, they believe they are informed.

\textbf{4. Control of the Agenda (Who decides what we discuss?)}
\textit{Grade:} \textbf{Fail.} Commercial algorithms and academic gatekeepers decide what topics are ``worthy'' of attention. SEO consultants and journal editors—not citizens—set the agenda.

\textbf{5. Inclusiveness (Is everyone invited?)}
\textit{Grade:} \textbf{Mixed.} The internet is open to all, but \textit{visibility} is not. You can publish, but without money or connections, you won't be found.

\hrulefill

\textbf{Bottom Line.}
Trust systems are not neutral. They are \textbf{designed mechanisms} with rules that can be gamed. Whether it's a citation ring inflating a scientist's reputation or a link farm boosting a website, the result is the same: the measure becomes the target. And when search engines can shift votes by 20\%—invisibly and personally—the line between ``ranking information'' and ``manipulating democracy'' disappears.

\newpage
\section*{Week 5: Election Methods}

\textbf{The Big Picture.}
We often think of Democracy as a set of values. This week teaches us that Democracy is an \textbf{Engineering Problem}. The specific code we use to count votes—the \textbf{Election Method}—determines the winner. How we count matters more than who votes.

\section*{Part 1: What Makes a Good Election System?}
Before choosing an algorithm, we need to know what we want from it. A good election system should have:

\textbf{Core Properties:}
\begin{itemize}[noitemsep, topsep=0pt, leftmargin=*]
    \item \textbf{Fairness:} Every vote should carry equal weight (Voter Fairness). The system should not favor specific candidates, e.g., based on ballot order (Candidate Fairness).
    \item \textbf{Integrity:} One person, one vote (no double counting). Secret ballots to prevent coercion or bribery.
    \item \textbf{Simplicity:} Easy to understand and execute for the average citizen.
    \item \textbf{Accessibility:} Voting must be accessible to everyone, including those with mobility challenges.
    \item \textbf{Determinism:} The same inputs should always produce the same result (no random winners).
    \item \textbf{Practicality:} The system should be logistically feasible and cost-effective.
\end{itemize}

\section*{Part 2: The Problem with Plurality}
The most common system (used in the US/UK) is \textbf{Plurality} (``First Past the Post'').
\textit{Rule:} Everyone gets one vote. The candidate with the most votes wins.

\textbf{Why it Fails:}
\begin{itemize}[noitemsep, topsep=0pt, leftmargin=*]
    \item \textbf{Vote Splitting:} If two similar candidates (e.g., Two Liberals vs One Conservative) run, they split the liberal vote. The Conservative wins, even if the majority of the country is Liberal.
    \item \textbf{Strategic Voting:} Voters are forced to vote for the ``lesser of two evils'' rather than their true preference.
    \item \textbf{Duverger's Law:} Because of vote splitting, voters are afraid to ``waste'' their vote on a third party. This mathematically forces a rigid \textbf{Two-Party System}.
\end{itemize}

\section*{Part 3: Better Algorithms (Alternative Voting)}
We can fix these bugs by changing the software of democracy.

\textbf{1. Runoff Voting (Two Rounds).}
If no candidate wins a majority ($>$50\%) in the first round, the top two face a second election.
\begin{itemize}[noitemsep, topsep=0pt, leftmargin=*]
    \item \textbf{Benefit:} Ensures the final winner has majority support.
    \item \textbf{Problem:} Expensive (running two elections) and causes \textbf{Voter Fatigue}—turnout often drops significantly in round two.
\end{itemize}

\textbf{2. Instant Runoff Voting (IRV) / Alternative Vote.}
Voters rank candidates. If no one has a majority, the candidate with the \textit{fewest} first-choice votes is eliminated, and their votes transfer to voters' next preferences. Repeat until a winner emerges.
\begin{itemize}[noitemsep, topsep=0pt, leftmargin=*]
    \item \textbf{Benefit:} Simulates a runoff in a single election event. This is the \textit{single-winner} version of STV.
\end{itemize}

\textbf{3. The Condorcet Method (Head-to-Head).}
Imagine a 1-on-1 race between every possible pair of candidates.
\begin{itemize}[noitemsep, topsep=0pt, leftmargin=*]
    \item \textbf{The Winner:} The candidate who beats \textit{everyone else} in these head-to-head match-ups.
    \item \textbf{Benefit:} Selects the ``Consensus Candidate'' that the broadest majority favors (often a Centrist).
    \item \textbf{Problem:} \textbf{Condorcet Cycles}—it's possible to have A beats B, B beats C, C beats A, with no clear winner.
\end{itemize}

\textbf{4. Borda Count (Ranked Scoring).}
Voters rank candidates (1st, 2nd, 3rd). 1st place gets max points, last gets 0.
\begin{itemize}[noitemsep, topsep=0pt, leftmargin=*]
    \item \textbf{Benefit:} Finds the candidate with the highest average approval, even if they aren't the passionate \#1 choice of any group.
\end{itemize}

\textbf{5. Coombs' Method (Eliminate the Most Hated).}
Similar to IRV, but instead of eliminating the candidate with the fewest first-place votes, it eliminates the candidate with the \textbf{most last-place votes}.
\begin{itemize}[noitemsep, topsep=0pt, leftmargin=*]
    \item \textbf{Benefit:} Eliminates polarizing candidates early.
\end{itemize}

\textbf{6. Approval Voting (Simplicity).}
Voters don't rank; they just ``approve'' (check the box) for as many candidates as they like.
\begin{itemize}[noitemsep, topsep=0pt, leftmargin=*]
    \item \textbf{Benefit:} Simple to count. Eliminates vote splitting (you can approve the Green Party AND the Democrat) so you never ``waste'' a vote.
\end{itemize}

\textbf{7. Single Transferable Vote (STV) – The Gold Standard.}
Voters rank candidates. A \textbf{Quota} (minimum votes to win) is calculated. If your \#1 choice is eliminated (has the fewest votes), your vote \textbf{transfers} to your \#2 choice. If your \#1 wins with extra votes (\textbf{Surplus}), those surplus votes also transfer.
\begin{itemize}[noitemsep, topsep=0pt, leftmargin=*]
    \item \textbf{No Wasted Votes:} You can vote for a small party without fear. If they lose, your vote still counts for your backup.
    \item \textbf{Proportional Representation:} If 30\% of the public supports the Green Party, they actually get $\sim$30\% of the seats.
    \item \textbf{Used In:} Ireland (national elections), Australia (Senate), Malta, Cambridge MA (USA).
\end{itemize}

\textbf{The Quota Problem (Hare vs. Droop).}
How many votes are needed to guarantee a seat? Two main formulas exist:
\begin{itemize}[noitemsep, topsep=0pt, leftmargin=*]
    \item \textbf{Hare Quota:} Total Votes / Seats. Simple, but can give unfair results.
    \item \textbf{Droop Quota:} (Total Votes / (Seats + 1)) + 1. The smallest number such that no more candidates than seats can reach it. More fair.
\end{itemize}


\section*{Part 4: The Scorecard (Dahl's Criteria)}
\textbf{Ranking the Voting Algorithms.}
\begin{itemize}[noitemsep, topsep=0pt, leftmargin=*]
    \item \textbf{Effective Participation:} \textbf{Fail (Plurality) / Pass (STV).} Under Plurality, millions of votes for losing candidates are ``wasted.'' Under STV, almost every vote helps elect someone.
    \item \textbf{Voting Equality:} \textbf{Mixed.} Gerrymandering (drawing district lines) rigs the game in Plurality systems. Proportional systems like STV make Gerrymandering impossible.
\end{itemize}

\textbf{Arrow's Impossibility Theorem.}
Kenneth Arrow proved mathematically that \textbf{no voting system is perfect}. Every system has some flaw (e.g., Condorcet can have ``Cycles''). Democracy is about choosing the system with the \textit{least bad} flaws.

\textbf{Gibbard-Satterthwaite Theorem.}
A related proof: \textbf{No ranking-based voting system can guarantee that voters always benefit from voting honestly.} Every system can be ``gamed'' in some scenario. The goal is to make gaming difficult, not impossible.

\textbf{Strategic vs. Honest Voting.}
\begin{itemize}[noitemsep, topsep=0pt, leftmargin=*]
    \item \textbf{Honest Voting:} You vote for your true preference.
    \item \textbf{Strategic Voting:} You vote for a less-preferred candidate because you think your true favorite can't win.
\end{itemize}
\textit{Key Insight:} Plurality \textit{forces} strategic voting (``don't waste your vote!''). Better systems like STV encourage honest voting because your backup choices still count.

\hrulefill

\textbf{Bottom Line.}
To fix democracy, we don't just need better candidates; we need better \textbf{Algorithms}. Replacing Plurality with \textbf{STV} or \textbf{Ranked Choice} would do more to fix polarization and representation than any single election result. But remember Arrow's law: perfection is impossible. The best we can do is choose the system with the least harmful trade-offs.


\newpage
\section*{Week 6: Blockchains, Smart Contracts, and DAOs}

\textbf{The Big Picture.}
This week explores the latest technological promise to ``democratize'' society: \textbf{Blockchain}. We examine Bitcoin, Ethereum, smart contracts, and Decentralized Autonomous Organizations (DAOs). We also explore \textbf{Liquid Democracy}—a hybrid voting system that blends direct and representative democracy. The key lesson: technology that \textit{promises} decentralization often \textit{delivers} new forms of centralization.

\section*{Part 1: The Three Waves of Decentralization}
Technology has promised to ``democratize everything'' multiple times. Each wave had idealistic beginnings and sobering outcomes.

\textbf{Wave 1: Usenet (1979).}
Global forums where anyone could post. Promised free speech for all. Outcome: Spam killed it.

\textbf{Wave 2: Peer-to-Peer (2000s).}
File sharing (Napster, BitTorrent). Promised to cut out middlemen. Outcome: Legal battles and centralized streaming (Spotify) won.

\textbf{Wave 3: Blockchain (2008–now).}
Bitcoin, Ethereum, DeFi, DAOs. Promises to cut out banks, lawyers, and governments. Outcome: Still unfolding, but early signs show familiar problems of centralization and manipulation.

\section*{Part 2: Bitcoin – The Technical Innovation}
Bitcoin, released in 2008 by the mysterious ``Satoshi Nakamoto,'' solved four hard computer science problems at once.

\textbf{1. The Agreement Problem.}
How do thousands of computers agree on a shared history without a boss? Bitcoin solved this with a ``blockchain''—a chain of records where each block references the previous one. Tampering with old records would break the chain.

\textbf{2. The Sybil Problem (Fake Identity Attack).}
On the internet, one person can create unlimited fake identities. In a vote, this means unlimited fake votes. Bitcoin made creating new ``identities'' (blocks) expensive by requiring computational work.

\textbf{3. Distributed Ledger.}
A shared document (the blockchain) that everyone can read and add to, but no one can edit or delete. It's ``write-only.''

\textbf{4. Digital Cash.}
The first truly self-sustaining digital currency. Unlike previous attempts, Bitcoin's incentive design made it valuable in the real world.

\section*{Part 3: Proof of Work – Elegant and Wasteful}
To add a block, miners must solve a cryptographic puzzle. This is called \textbf{Proof of Work}.

\textbf{Why It Works:}
\begin{itemize}[noitemsep, topsep=0pt, leftmargin=*]
    \item \textbf{Rate Limiting:} Slows down block creation, giving time for the network to synchronize.
    \item \textbf{Skin in the Game:} Miners who invest resources are incentivized to keep the system honest.
    \item \textbf{Permissionless Entry:} Originally, anyone could join without asking permission.
\end{itemize}

\textbf{Why It Fails:}
\begin{itemize}[noitemsep, topsep=0pt, leftmargin=*]
    \item \textbf{Energy Waste:} The work is \textit{intentionally useless}. Bitcoin consumes more electricity than many countries. It's a ``climate disaster of a way to make money.''
    \item \textbf{Centralization:} Mining requires specialized hardware, cheap electricity, and capital. The ``permissionless'' ideal failed—mining is now controlled by a few industrial operations.
    \item \textbf{Alternatives Exist:} \textbf{Proof of Stake} and other mechanisms avoid waste. New blockchains should not use Proof of Work.
\end{itemize}

\section*{Part 4: Smart Contracts – Code as Law}
Bitcoin allowed small programs (``scripts'') to be attached to transactions. Ethereum expanded this into full \textbf{Smart Contracts}—programs that run on the blockchain.

\textbf{How Smart Contracts Work:}
\begin{itemize}[noitemsep, topsep=0pt, leftmargin=*]
    \item A smart contract is a program stored on the blockchain.
    \item When triggered, every computer in the network runs the same code.
    \item The output is recorded on the blockchain forever.
    \item Example: A ``multi-signature wallet'' requires 2 out of 3 key holders to approve a transaction.
\end{itemize}

\textbf{Fundamental Limitations:}
\begin{itemize}[noitemsep, topsep=0pt, leftmargin=*]
    \item \textbf{Determinism Required:} Every computer must get the exact same answer. No randomness allowed.
    \item \textbf{No External Data:} Contracts cannot access websites or the outside world (different computers might see different data, breaking consensus).
    \item \textbf{Execution Limits:} Ethereum uses ``Gas'' (a fee) to limit how long a program can run. This prevents denial-of-service attacks.
\end{itemize}

\section*{Part 5: DAOs – The Dream of Algorithmic Governance}
A \textbf{Decentralized Autonomous Organization (DAO)} is an organization whose rules are written entirely in code.

\textbf{The Promise:}
\begin{itemize}[noitemsep, topsep=0pt, leftmargin=*]
    \item \textbf{Automatic Enforcement:} Rules execute exactly as coded—no human discretion or bias.
    \item \textbf{Transparency:} All code and transactions are public. Anyone can audit.
    \item \textbf{``Code is Law'':} Whatever the code says, happens. No lawyers, no judges.
\end{itemize}

\textbf{The Reality (Problems):}
\begin{itemize}[noitemsep, topsep=0pt, leftmargin=*]
    \item \textbf{Bugs are Legal Bugs:} If the code has a security flaw, exploiting it is technically ``following the rules.''
    \item \textbf{Upgrade Dilemma:} To fix bugs, DAOs need upgrade mechanisms. But whoever controls the upgrade key has centralized power—defeating the point.
    \item \textbf{Hidden Centralization:} Many ``decentralized'' projects are controlled by small development teams who hold the keys.
\end{itemize}

\section*{Part 6: Case Study – ``The DAO'' Hack (2016)}
``The DAO'' was an Ethereum-based investment fund that raised \textbf{\$250 million USD}—about 14\% of all Ethereum at the time.

\textbf{The Attack.}
An anonymous hacker found a ``recursion bug'' in the code and began draining funds. The code allowed it—technically, nothing was ``broken.''

\textbf{The Debate.}
\begin{itemize}[noitemsep, topsep=0pt, leftmargin=*]
    \item \textbf{``Code is Law'' Purists:} The attacker exploited public code. That's how the system is \textit{supposed} to work. No intervention.
    \item \textbf{Pragmatists:} Investors don't want to lose money. If The DAO fails, Ethereum's reputation (and value) collapses with it.
\end{itemize}

\textbf{The Resolution: ``Too Big to Fail.''}
Ethereum's developers performed a \textbf{Hard Fork}—rewriting the blockchain to undo the hack, return the funds, and shut down The DAO. This was the blockchain equivalent of a government bailout.

\textbf{The Split: Ethereum Classic (ETC).}
A minority refused to accept the intervention. They continued the original chain (where the hack ``succeeded'') as \textbf{Ethereum Classic}—a philosophical protest against centralized override.

\textbf{Key Insight.}
The DAO proved that ``algorithmic authority'' is not absolute. When the code fails catastrophically, humans intervene with ``practical governance.'' Technology cannot easily replace centuries of refined human institutions.

\section*{Part 7: Liquid Democracy – A Hybrid Voting System}
\textbf{Liquid Democracy} is an alternative to both direct and representative democracy.

\textbf{How It Works:}
\begin{itemize}[noitemsep, topsep=0pt, leftmargin=*]
    \item You can vote directly on any issue.
    \item Or you can \textbf{delegate} your vote to someone you trust (a ``proxy'').
    \item Delegations are \textbf{transitive}: If Alice delegates to Bob, and Bob delegates to Carol, then Carol controls Alice's vote.
    \item You can \textbf{override} or revoke your delegation at any time.
\end{itemize}

\textbf{The Golden Rule.}
``If I give you my vote, I can see what you do with it.'' Transparency is essential for accountability.

\textbf{Real-World Experiments:}
\begin{itemize}[noitemsep, topsep=0pt, leftmargin=*]
    \item \textbf{LiquidFeedback (Germany's Pirate Party):} Used from 2009–2013 for internal party decisions.
    \item \textbf{Google Votes (2012–2015):} An internal Google experiment. Over 20,000 employees cast 87,000 votes on 370 issues (mostly food and office decisions).
\end{itemize}

\section*{Part 8: Super-Voters – The Power Law Returns}
A study of LiquidFeedback revealed a familiar pattern: \textbf{Power Laws}.

\textbf{The Finding.}
A small number of users (``super-voters'') accumulated a huge share of delegations—just like Twitter followers or PageRank.

\textbf{The Surprise: Super-Voters as Stabilizers.}
Contrary to fears, super-voters did \textit{not} abuse their power:
\begin{itemize}[noitemsep, topsep=0pt, leftmargin=*]
    \item They voted with the majority more often than average users.
    \item They felt ``social pressure'' to use their weight responsibly.
    \item They acted as a \textbf{stabilizing force}, preventing deadlock.
\end{itemize}

\textbf{Why Did Delegation Happen?}
Surprisingly, people did \textit{not} delegate based on shared political views (agreement rates were only slightly above random). Delegation seems driven by trust and perceived competence, not ideology.

\section*{Part 9: The Scorecard (Dahl's Criteria)}
\textbf{How Do Blockchain/DAO/Liquid Democracy Systems Measure Up?}

\textbf{1. Effective Participation (Can you be heard?)}
\textit{Grade:} \textbf{Mixed.} In theory, anyone can participate. In practice, mining is industrialized, and DAO governance is often controlled by large token holders (``plutocracy by coin'').

\textbf{2. Voting Equality (One Person, One Vote).}
\textit{Grade:} \textbf{Fail.} Most blockchain voting is weighted by how many tokens you own. Rich actors have more votes. Liquid Democracy's super-voters also concentrate power.

\textbf{3. Enlightened Understanding (Do you know the truth?)}
\textit{Grade:} \textbf{Mixed.} Code is public and transparent—if you can read it. But understanding smart contracts requires expertise. Most users cannot audit the code they trust.

\textbf{4. Control of the Agenda (Who decides what we discuss?)}
\textit{Grade:} \textbf{Fail.} In DAOs, developers propose upgrades. In Liquid Democracy, super-voters shape outcomes. Ordinary users rarely set the agenda.

\textbf{5. Inclusiveness (Is everyone invited?)}
\textit{Grade:} \textbf{Fail.} Blockchain participation requires hardware, electricity, capital, and technical knowledge. The ``permissionless'' ideal gave way to industrial gatekeeping.

\hrulefill

\textbf{Bottom Line.}
Blockchain promised to replace trust in institutions with trust in code. But \textbf{code is written by humans}—and humans have bugs, biases, and interests. The DAO showed that when stakes are high, communities revert to human governance (hard forks, bailouts). Liquid Democracy offers a flexible alternative to rigid elections, but it too concentrates power in super-voters. The core lesson: \textbf{decentralization is a process, not a product.} No technology can permanently solve the problem of power—it can only redistribute it.


\newpage
\section*{Week 7: Liquid Democracy, Prediction Markets, and Quadratic Voting}

\textbf{The Big Picture.}
This week explores three innovative approaches to democratic decision-making: \textbf{Liquid Democracy} (a flexible hybrid of direct and representative democracy), \textbf{Prediction Markets} (using market prices to aggregate information), and \textbf{Quadratic Voting} (letting voters express how \textit{much} they care). Each technology addresses a specific weakness in traditional democracy. The core question: \textit{Can we design better systems for collective decision-making?}

\section*{Part 1: The Democracy Dilemma (Direct vs. Representative)}

Traditional democracy forces a choice between two imperfect systems.

\textbf{Representative Democracy.}
You elect someone to vote on your behalf.
\begin{itemize}[noitemsep, topsep=0pt, leftmargin=*]
    \item \textbf{Pro:} Saves you time. Politicians can specialize in understanding complex issues.
    \item \textbf{Con:} Once elected, representatives may not represent \textit{you}. You're stuck with their choices on every issue.
    \item \textbf{Con:} Limited choice—you pick a party that matches you on \textit{some} issues, but not all.
\end{itemize}

\textbf{Direct Democracy.}
You vote on every issue yourself.
\begin{itemize}[noitemsep, topsep=0pt, leftmargin=*]
    \item \textbf{Pro:} Maximum control. Your voice counts on every decision.
    \item \textbf{Con:} Exhausting. Who has time to research every issue?
    \item \textbf{Con:} ``Hot-button'' issues dominate attention while important but boring issues get ignored.
\end{itemize}

\textbf{The Swiss Model.}
Switzerland shows direct democracy \textit{can} work: decentralized power, balanced information packets sent to voters, and a strong culture of participation. But execution quality matters enormously.

\section*{Part 2: Liquid Democracy – The Hybrid Solution}

Liquid Democracy (also called \textbf{Delegative Democracy}) offers a ``best of both worlds'' approach.

\textbf{The Core Idea.}
For each issue, you choose:
\begin{itemize}[noitemsep, topsep=0pt, leftmargin=*]
    \item \textbf{Vote directly} on topics you care about or understand well.
    \item \textbf{Delegate your vote} to someone you trust for topics you don't want to engage with.
\end{itemize}

\textbf{Key Innovation: Issue-by-Issue Flexibility.}
Unlike traditional elections (one representative for everything), liquid democracy lets you:
\begin{itemize}[noitemsep, topsep=0pt, leftmargin=*]
    \item Delegate to a doctor friend on healthcare issues.
    \item Delegate to an economist colleague on tax policy.
    \item Vote directly on education (because you're a teacher).
    \item Change or revoke delegation at any time.
\end{itemize}

\textbf{Historical Background.}
\begin{itemize}[noitemsep, topsep=0pt, leftmargin=*]
    \item \textbf{Lewis Carroll (1800s):} First proposed ``vote clubbing'' (combining votes).
    \item \textbf{Corporate Boardrooms:} Proxy voting has been standard practice for over a century.
    \item \textbf{2003:} The term ``liquid democracy'' was coined.
    \item \textbf{German Pirate Party:} Major experiment using \textbf{LiquidFeedback} platform (2009–2013).
\end{itemize}

\section*{Part 3: Design Choices in Liquid Democracy}

Building a liquid democracy system requires answering hard questions.

\textbf{1. Should Delegation Be Transitive?}

If Alice delegates to Bob, and Bob delegates to Carol—does Carol control Alice's vote?

\begin{itemize}[noitemsep, topsep=0pt, leftmargin=*]
    \item \textbf{Against:} I trust you to vote for me, not to \textit{choose who votes for me}. Transitive chains can lead to unexpected power concentration.
    \item \textbf{For:} Trust chains work naturally. You trust your neighbor; your neighbor trusts an expert. This creates an ``information ladder'' to expertise.
    \item \textbf{Compromise:} Allow ``ranked-choice delegation'' (if your first choice doesn't vote, fall back to second choice) or limit chain length.
\end{itemize}

\textbf{2. The Super-Voter Problem.}

What if everyone delegates to the same popular person (a celebrity, expert, or charismatic leader)?

\begin{itemize}[noitemsep, topsep=0pt, leftmargin=*]
    \item \textbf{Risk:} Creates an ``accidental dictator'' with concentrated power.
    \item \textbf{German Pirate Party Example:} One professor accumulated massive delegated power—``the most powerful pirate.''
    \item \textbf{Potential Fix: Split Delegation.} Allow voters to divide their vote among multiple delegates (e.g., 50\% to Alice, 30\% to Bob, 20\% to Carol). This spreads power more evenly.
\end{itemize}

\textbf{3. Privacy and Coercion.}

In traditional voting, secrecy prevents vote buying. But in liquid democracy, delegates must be visible for accountability.

\begin{itemize}[noitemsep, topsep=0pt, leftmargin=*]
    \item \textbf{Solution 1—Thresholds:} Require a minimum number of delegations before someone can act as a delegate. Harder to buy one vote if you need 100.
    \item \textbf{Solution 2—Public/Private Separation:} Your \textbf{public vote} (as a delegate) is visible for accountability. Your \textbf{private vote} (as a citizen) remains secret.
\end{itemize}

\textbf{4. Delegation Cycles.}

What if Alice delegates to Bob, Bob to Carol, and Carol back to Alice? Votes loop forever.

\begin{itemize}[noitemsep, topsep=0pt, leftmargin=*]
    \item \textbf{Solution (``Viscous Democracy''):} Lose a small fraction of voting power at each hop. Cycles naturally decay to zero.
\end{itemize}

\section*{Part 4: Real-World Experiments}

\textbf{Experiment 1: LiquidFeedback (German Pirate Party, 2009–2013).}
\begin{itemize}[noitemsep, topsep=0pt, leftmargin=*]
    \item Used for internal party decisions.
    \item \textbf{Result:} Super-voters emerged (power law distribution), but surprisingly did \textit{not} abuse power.
    \item Super-voters voted with the majority more often than average users.
    \item They felt ``social pressure'' to be responsible—acting as \textbf{stabilizers}, not dictators.
    \item Delegation was \textit{not} based on shared political views—it seemed driven by trust and perceived competence.
\end{itemize}

\textbf{Experiment 2: Google Votes (2012–2015).}
\begin{itemize}[noitemsep, topsep=0pt, leftmargin=*]
    \item Internal Google experiment on corporate Google+ network.
    \item Over 3 years: 20,000 employees, 87,000 votes, 370 issues.
    \item Primary use: Low-stakes decisions (office snacks, cafeteria menus, T-shirt designs).
    \item \textbf{Only 3.6\% of votes were delegated}—most people voted directly on accessible topics.
    \item Niche experts (e.g., someone with vegan diet expertise) successfully attracted delegations in their specialty.
    \item Supported multiple voting methods: Yes/No, Approval, Score (1-5 stars), and Ranked (using Schulze method).
\end{itemize}

\textbf{The Golden Rule of Liquid Democracy.}
``If I give you my vote, I can see what you do with it.'' Transparency enables accountability.

\section*{Part 5: Prediction Markets – Betting on the Truth}

\textbf{Prediction Markets} take a completely different approach: use \textit{money} to aggregate information.

\textbf{How They Work.}
\begin{itemize}[noitemsep, topsep=0pt, leftmargin=*]
    \item A market trades contracts that pay out based on future events.
    \item Example: A contract pays \$1 if Candidate X wins the election.
    \item If the contract trades at 53 cents, the market ``believes'' there's a \textbf{53\% chance} Candidate X wins.
\end{itemize}

\textbf{Why They Work.}
\begin{itemize}[noitemsep, topsep=0pt, leftmargin=*]
    \item \textbf{Skin in the Game:} Unlike polls (where lying is free), betting real money forces people to be honest about what they believe.
    \item \textbf{Information Aggregation:} Markets combine the knowledge of thousands of participants into a single price.
    \item \textbf{Continuous Updates:} Prices update instantly as new information arrives—unlike polls conducted weekly.
\end{itemize}

\textbf{Accuracy: Better Than Polls.}
The Iowa Electronic Markets tracked US presidential elections:
\begin{itemize}[noitemsep, topsep=0pt, leftmargin=*]
    \item \textbf{Week before election:} Prediction market error = 1.5\%. Gallup Poll error = 2.1\%.
    \item \textbf{150 days before election:} Markets far more accurate than polls.
\end{itemize}

\textbf{Real-World Applications.}
\begin{itemize}[noitemsep, topsep=0pt, leftmargin=*]
    \item \textbf{US Government:} National security threat assessments.
    \item \textbf{Healthcare:} Predicting flu outbreak severity.
    \item \textbf{Corporations:} Google, Intel, Microsoft, Eli Lilly use internal prediction markets for product launch timing, sales forecasts, and software quality.
\end{itemize}

\textbf{The Legal Problem.}
Prediction markets look like gambling. US law restricts them heavily:
\begin{itemize}[noitemsep, topsep=0pt, leftmargin=*]
    \item Eight states ban Internet gambling outright.
    \item The Unlawful Internet Gambling Enforcement Act (2006) creates legal ambiguity.
    \item Proposed solution: Create ``safe harbors'' for small-stakes, research-focused markets.
\end{itemize}

\section*{Part 6: Quadratic Voting – Expressing Intensity}

Traditional voting has a fundamental problem: my passionate ``Yes'' counts the same as your indifferent ``Yes.''

\textbf{Quadratic Voting (QV)} solves this by letting voters express \textit{how much} they care.

\textbf{The Mechanism.}
\begin{itemize}[noitemsep, topsep=0pt, leftmargin=*]
    \item Each voter receives a budget of ``voice credits.''
    \item You can buy votes on any issue, but the cost is \textbf{quadratic}: \\
    1 vote = 1 credit. 2 votes = 4 credits. 3 votes = 9 credits. 10 votes = 100 credits.
    \item This creates a trade-off: spend heavily on issues you care deeply about, or spread influence across many issues.
\end{itemize}

\textbf{Why Quadratic? (Not Linear, Not Cubic)}
\begin{itemize}[noitemsep, topsep=0pt, leftmargin=*]
    \item \textbf{Linear cost} (1 vote = 1 credit) leads to \textbf{dictatorship}: whoever cares most buys infinite votes.
    \item \textbf{Very high cost} (votes become impossible to buy) leads to \textbf{1-person-1-vote}—back to ignoring intensity.
    \item \textbf{Quadratic is the sweet spot:} votes become proportional to \textit{how much you care}. Mathematically proven to be uniquely optimal.
\end{itemize}

\textbf{The Problem QV Solves: Tyranny of the Majority.}
Under traditional voting, a \textit{mildly interested} majority defeats a \textit{passionate} minority—even when the minority cares far more.

\begin{itemize}[noitemsep, topsep=0pt, leftmargin=*]
    \item Example: 60\% mildly prefer Option A. 40\% desperately need Option B.
    \item Traditional voting: A wins (60\% vs 40\%).
    \item Quadratic voting: The 40\% can ``outspend'' the apathetic 60\% by buying more votes.
\end{itemize}

\textbf{Evidence.}
\begin{itemize}[noitemsep, topsep=0pt, leftmargin=*]
    \item Lab experiments show QV produces decisions much closer to optimal than 1-person-1-vote.
    \item Already used in polling and survey research with significant improvements over traditional methods.
    \item Higher-stakes applications (political elections, corporate governance) require more experimentation.
\end{itemize}

\textbf{``Radical Democracy'' – Three Meanings.}
\begin{itemize}[noitemsep, topsep=0pt, leftmargin=*]
    \item \textbf{Mathematical:} Votes equal the \textit{square root} (radical) of credits spent.
    \item \textbf{Philosophical:} Returns to original democratic ideals—maximizing the ``general happiness.''
    \item \textbf{Liberating:} Frees citizens from the ``straightjacket'' of one-person-one-vote rationing.
\end{itemize}

\section*{Part 7: The Scorecard (Dahl's Criteria)}

\textbf{How Do These Innovations Measure Up?}

\textbf{1. Effective Participation (Can you be heard?)}
\begin{itemize}[noitemsep, topsep=0pt, leftmargin=*]
    \item \textbf{Liquid Democracy:} \textbf{Pass.} You can participate as much or as little as you want.
    \item \textbf{Prediction Markets:} \textbf{Mixed.} You need money to participate. Richer participants have more influence.
    \item \textbf{Quadratic Voting:} \textbf{Pass.} Everyone gets voice credits; intensity of preference is heard.
\end{itemize}

\textbf{2. Voting Equality (One Person, One Vote).}
\begin{itemize}[noitemsep, topsep=0pt, leftmargin=*]
    \item \textbf{Liquid Democracy:} \textbf{Mixed.} Super-voters accumulate power through delegation.
    \item \textbf{Prediction Markets:} \textbf{Fail.} Voting power is proportional to wealth.
    \item \textbf{Quadratic Voting:} \textbf{Pass (if credits are equal).} Equal starting budgets maintain fairness.
\end{itemize}

\textbf{3. Enlightened Understanding (Do you know the truth?)}
\begin{itemize}[noitemsep, topsep=0pt, leftmargin=*]
    \item \textbf{Liquid Democracy:} \textbf{Mixed.} You can delegate to experts, but you're trusting their judgment.
    \item \textbf{Prediction Markets:} \textbf{Pass.} Markets aggregate dispersed knowledge effectively.
    \item \textbf{Quadratic Voting:} \textbf{Neutral.} Doesn't address information quality.
\end{itemize}

\textbf{4. Control of the Agenda (Who decides what we discuss?)}
\begin{itemize}[noitemsep, topsep=0pt, leftmargin=*]
    \item \textbf{Liquid Democracy:} \textbf{Mixed.} Super-voters may shape outcomes; ordinary users rarely set agenda.
    \item \textbf{Prediction Markets:} \textbf{Mixed.} Market creators set which questions to ask.
    \item \textbf{Quadratic Voting:} \textbf{Neutral.} Works on given issues; doesn't determine which issues are raised.
\end{itemize}

\textbf{5. Inclusiveness (Is everyone invited?)}
\begin{itemize}[noitemsep, topsep=0pt, leftmargin=*]
    \item \textbf{Liquid Democracy:} \textbf{Pass.} Low barriers if implemented well.
    \item \textbf{Prediction Markets:} \textbf{Fail.} Requires capital and market access.
    \item \textbf{Quadratic Voting:} \textbf{Pass.} Equal credit distribution enables broad participation.
\end{itemize}

\hrulefill

\textbf{Bottom Line.}
Democracy is an engineering problem. Traditional one-person-one-vote is not the only solution—and may not be the best. \textbf{Liquid Democracy} lets you choose when to participate and when to delegate, matching your engagement to your expertise. \textbf{Prediction Markets} harness financial incentives to aggregate information more accurately than polls. \textbf{Quadratic Voting} lets you express not just \textit{what} you want, but \textit{how much} you want it. None of these is perfect. Liquid Democracy still creates power concentrations. Prediction Markets favor the wealthy. Quadratic Voting requires careful credit distribution. But each represents a genuine innovation—a new tool in the democratic toolkit. The lesson: \textbf{democracy can evolve with technology}, but only if we design carefully, experiment honestly, and remain vigilant about where power actually flows.

\end{document}
