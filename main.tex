\documentclass[10pt, a4paper]{article}
\usepackage[margin=0.7in]{geometry}
\usepackage{enumitem}
\usepackage{titlesec}
\usepackage{parskip}
\usepackage{tabularx}
\usepackage{booktabs}
\usepackage{amsmath}
\usepackage[activate={true,nocompatibility},final,tracking=true,kerning=true,spacing=true,factor=1100,stretch=10,shrink=10]{microtype}

% Prevent hyphenation (cut-off words)
\tolerance=1
\emergencystretch=\maxdimen
\hyphenpenalty=10000
\hbadness=10000

% Prevent widows and orphans
\widowpenalty=10000
\clubpenalty=10000

% Compact title
\titleformat{\section}{\large\bfseries}{\thesection}{1em}{}
\titlespacing*{\section}{0pt}{1.5ex plus 0.5ex minus .2ex}{0.5ex plus .2ex}
\setlength{\parindent}{0pt}
\setlength{\parskip}{0.6em}

\begin{document}
\begin{titlepage}
    \centering
    \vspace*{\fill}
    {\Huge \textbf{Technologies for Democratic Societies}\par}
    \vspace*{\fill}
\begin{abstract}
\noindent\textbf{Attention!} All content in this summary was generated with the use of AI. Please double-check all information for accuracy before relying on it.
\end{abstract}
    \vspace*{\fill}
\end{titlepage}


\newpage
\section*{Week 1 \& 2: The Vision \& The Experiment}

\textbf{The Big Picture.}
This course asks a simple question: \textit{Does technology help democracy or hurt it?}
We don't just ask "how do we build it?" but "what should we build?" We look at the gap between the \textbf{dream} (everyone has a voice) and the \textbf{reality} (spam, fake news, and surveillance).

\section*{Part 1: The Dream (Licklider's Vision, 1968)}
Long before the internet, J.C.R. Licklider dreamed of a future where computers would help people think and work together better.

\textbf{Communication is "Modeling," not just Sending.}
Licklider argued that real communication isn't just sending a file to someone. It's about properly understanding what is in the other person's head (their "mental model"). He believed computers would help us show these models to each other, making teamwork faster and smarter than face-to-face meetings.

\textbf{The "OLIVER" (Digital Assistant).}
He predicted we would all have a digital assistant (he called it an OLIVER) that knows what we like and helps us filter information. Today, these are the algorithms and AI bots we interact with.

\textbf{Communities based on Interest, not Location.}
He predicted that people would form groups based on what they \textit{love}, not just where they \textit{live}. This would make life happier and more productive.

\textbf{The Warning: The Digital Divide.}
He also worried: If only rich people have computers, will this new power be a privilege for a few, or a right for everyone?

\section*{Part 2: The Experiment (Usenet, 1979)}
Usenet was the first real attempt to build this dream. It was the "grandfather" of Reddit and Twitter -- a global network where anyone could post a message for the world to see.

\textbf{How it worked (Simple Version):}
\begin{itemize}[noitemsep, topsep=0pt, leftmargin=*]
    \item \textbf{UUCP (Copying Files):} It started with a simple trick. One computer would call another over a phone line and "copy" a file to it.
    \item \textbf{Gossip Protocol:} That computer would then call another, and another. Like a rumor spreading in a school, eventually, every computer had the message.
    \item \textbf{Newsgroups:} Instead of emailing one person ("To: Bob"), you posted to a topic ("To: Unix-Lovers").
\end{itemize}

\textbf{The Philosophy: No Bosses, No Censorship.}
Usenet was \textbf{Decentralized}. No single company owned it. It was run by regular people (admins) connecting their computers.
It was designed to be \textbf{Censorship Resistant}. If one computer blocked a message, the "gossip" would just find another path around it. This meant total Freedom of Speech.

\textbf{Why it Failed: The Spam Explosion.}
Because the system was designed to never block anyone, it couldn't block bad actors.
\begin{itemize}[noitemsep, topsep=0pt, leftmargin=*]
    \item \textbf{Limitless Spam:} In 1994, lawyers Canter \& Siegel posted an ad for "Green Cards" to every single group. This proved you could use Usenet for free advertising.
    \item \textbf{Signal-to-Noise Collapse:} Soon, Usenet was full of junk. It became too hard to find the real conversations. People left for places like Reddit where moderators could delete the junk.
\end{itemize}

\section*{Part 3: The Scorecard (Dahl's Criteria)}
How do we judge if a technology like Usenet is actually "democratic"? We use Robert Dahl's 5 rules for a democracy.

\textbf{The Base Rule: We are all Competent.}
Democracy assumes that every adult is smart enough to decide what is best for themselves.

\textbf{1. Effective Participation (Can you be heard?)}
\textit{Rule:} Everyone must have a real chance to speak and be heard \textit{before} a decision is made.
\textit{Usenet's Grade:} \textbf{Fail.} Technically everyone could speak, but because of all the spam/noise, nobody could actually be heard.

\textbf{2. Voting Equality (One Person, One Vote).}
\textit{Rule:} When it's time to decide, my vote and your vote must count exactly the same.
\textit{Usenet's Grade:} \textbf{Fail.} Because there were no ID checks, one person could make 100 fake accounts ("Sock Puppets") and vote 100 times.

\textbf{3. Enlightened Understanding (Do you know the truth?)}
\textit{Rule:} You need access to the truth and different facts to make a smart choice.
\textit{Usenet's Grade:} \textbf{Mixed.} It promised access to all human knowledge, but without filters, it was easy to get lost in lies or junk data.

\textbf{4. Control of the Agenda (Who decides what we discuss?)}
\textit{Rule:} The people (not a king or a CEO) must decide what topics are important.
\textit{Usenet's Grade:} \textbf{Good start, poor finish.} At first, anyone could start a topic. Later, spammers hijacked the attention, effectively controlling what everyone saw.

\textbf{5. Inclusiveness (Is everyone invited?)}
\textit{Rule:} Every adult subject to the rules must be allowed to join.
\textit{Usenet's Grade:} \textbf{Pass.} It was famously open. If you could get to a computer, you were in.

\hrulefill

\textbf{Bottom Line.}
Usenet proves that \textbf{Uncensored Speech} (freedom to say anything) is not the same as \textbf{Democracy} (freedom to decide together). Without rules to stop abuse (like spam or fake voters), the democratic process breaks down.


\newpage
\section*{Week 3: The Metric of Influence}

\textbf{The Big Picture.}
This week asks: \textit{What is influence, and how do we measure it?} We look at three ways influence is measured today: \textbf{Social} (Twitter), \textbf{Algorithmic} (Google), and \textbf{Economic} (Wealth). A key claim: \textbf{Democracy itself is an influence metric.} Voting is how we measure who convinced whom. Campaigning is the attempt to \textit{create} influence; the vote is the \textit{measurement} of its success.

\textbf{The Danger:} These systems follow \textbf{Power Laws} (the "Rich get Richer"). A tiny number of people get almost all the attention, leaving everyone else invisible.

\section*{Part 1: Social Influence (The "Million Follower" Fallacy)}
We assume "more followers = more influence." A study of 54 million Twitter users proves this is a \textbf{Fallacy}.

\textbf{Three Measures of Influence:}
\begin{itemize}[noitemsep, topsep=0pt, leftmargin=*]
    \item \textbf{Indegree (Followers):} Measures \textbf{Popularity}. People are watching, but not necessarily listening. (Top users: CNN, Obama, Britney Spears)
    \item \textbf{Retweets:} Measures \textbf{Content Value}. People share what you say. (Top users: Mashable, Guy Kawasaki, NYTimes)
    \item \textbf{Mentions:} Measures \textbf{Name Value/Engagement}. People talk \textit{about} you. (Top users: Celebrities like Ashton Kutcher)
\end{itemize}
\textit{Key Finding:} The top 20 lists for each metric had almost \textbf{no overlap} (only 2 users in common). Popularity $\neq$ Influence.

\textbf{Two Theories of Influence:}
\begin{itemize}[noitemsep, topsep=0pt, leftmargin=*]
    \item \textbf{Traditional ("Influentials" Theory):} A small elite of "opinion leaders" drives trends. Target them, and the message spreads.
    \item \textbf{Modern ("Accidental Influentials" Theory):} Trends depend on how \textit{ready} society is to adopt them, not on who starts them. Anyone can spark a trend if the timing is right.
\end{itemize}
\textit{The Data's Verdict:} The traditional view is mostly right. Influence is \textbf{not gained by accident, but through concerted effort}—consistently posting valuable content on a single topic. The top influentials dominated across many different topics.

\textbf{The Streisand Effect.} Trying to censor something can backfire. When Barbra Streisand tried to stamp out negative content about herself, it only drew more attention to it. Retweets can spread ideas you hate just as easily as ideas you love.

\section*{Part 2: Algorithmic Influence (PageRank \& Google)}
Before Google, search engines like AltaVista just counted keywords. Results were random and useless. Google fixed this by treating \textbf{Links as Votes}.

\textbf{How PageRank Works:}
\begin{itemize}[noitemsep, topsep=0pt, leftmargin=*]
    \item \textbf{Recursive Importance:} A page is important if \textit{important pages} link to it. One link from the New York Times is worth more than 1,000 links from random blogs.
    \item \textbf{The "Rank Sink" Problem:} A group of pages linking only to each other would hoard all the power like dictators.
    \item \textbf{The Solution (Random Surfer / UBI):} The algorithm imagines a web surfer who clicks randomly but occasionally "gets bored" and jumps to a completely random page. This gives every page a tiny baseline chance—a "Universal Basic Income" of attention—preventing total monopoly.
\end{itemize}

\textbf{Personalized PageRank \& Filter Bubbles.}
The "random jump" destination (the E vector) can be customized. If E is uniform, it's a democratic view of the web. If E is your bookmarks, it's a \textit{personalized} view. This is powerful but risks creating \textbf{Filter Bubbles} where you only see what you already agree with.

\textbf{Manipulation: SEO \& Google Bombing.}
PageRank is \textit{harder} to game than simple keyword counting, but not impossible.
\begin{itemize}[noitemsep, topsep=0pt, leftmargin=*]
    \item \textbf{SEO (Search Engine Optimization):} The art of gaming the algorithm. Edit Wikipedia, build link farms, etc.
    \item \textbf{Google Bombing:} A group coordinates to link a specific word to a specific page. Example: Activists made a political candidate's name the top search result for an offensive made-up word.
\end{itemize}

\textbf{The Pornography Paradox.}
PageRank revealed something interesting: Pornographic sites had very high \textit{usage} (from web traffic data) but very low \textit{PageRank}. Why? People consume them privately but don't link to them publicly. PageRank measures what people \textit{endorse}, not just what they \textit{consume}.

\section*{Part 3: Economic Influence (Piketty's Inequality)}
Economist Thomas Piketty analyzed 100+ years of tax data and found a \textbf{U-shaped Curve} of inequality.

\textbf{The Core Finding:}
\begin{itemize}[noitemsep, topsep=0pt, leftmargin=*]
    \item \textbf{The Anomaly:} The relative equality of the mid-20th century (1940s-1970s) was a historical accident caused by wars and policy, not the natural state of capitalism.
    \item \textbf{The Return to Inequality:} Since 1980, wealth has concentrated sharply at the top (Top 1\% now own $>$33\% of U.S. wealth). The U.S. is now \textit{more unequal than Europe}—a complete reversal from 100 years ago.
    \item \textbf{The Formula ($r > g$):} When the \textbf{return on capital (r)} (stocks, real estate) is greater than \textbf{economic growth (g)}, wealth concentrates. Those who own things get richer faster than those who work.
\end{itemize}
\textit{Why it Matters for Democracy:} In a world where money buys influence (ads, campaigns, lobbying), extreme inequality breaks the promise of "One Person, One Vote."

\section*{Part 4: The Scorecard (Dahl's Criteria)}
Does the modern "Influence Economy" help or hurt democracy?

\textbf{1. Effective Participation (Can you be heard?)}
\textit{Grade:} \textbf{Fail.} Power Laws mean the top 1\% of accounts get almost all the attention. If you are new, it's nearly impossible to break through.

\textbf{2. Voting Equality (One Person, One Vote).}
\textit{Grade:} \textbf{Fail.} PageRank explicitly gives more voting power to "important" pages. Wealth inequality gives more power to the rich in real elections.

\textbf{3. Enlightened Understanding (Do you know the truth?)}
\textit{Grade:} \textbf{Pass (with caveats).} Google made finding facts easier than ever. But Personalized PageRank and algorithms can trap you in Filter Bubbles.

\textbf{4. Control of the Agenda (Who decides what we discuss?)}
\textit{Grade:} \textbf{Mixed.} Algorithms (and those who game them via SEO/Google Bombing) increasingly decide what news you see. The agenda is set by code, not citizens.

\hrulefill

\textbf{Bottom Line.}
Technology has given us powerful tools to measure influence (Twitter metrics, PageRank, wealth data). But these tools consistently show \textbf{Power Laws}—a few get almost everything. A search engine is not a democracy. Influence is not equally distributed. And the gap between who \textit{speaks} and who is \textit{heard} remains the central challenge for democratic technology.


\newpage
\section*{Week 4: Trust and Online Reputation}

\textbf{The Big Picture.}
This week asks a fundamental question: \textit{How do we design systems that aggregate millions of individual inputs—whether citations, hyperlinks, or votes—into a single, trustworthy collective outcome?}

In a large democracy, we cannot personally know the reputation of every expert or the quality of every website. Instead, we rely on \textbf{Proxy Systems} to measure value for us: \textbf{Peer Review} for scientific truth, \textbf{Search Engines} for information relevance, and \textbf{Election Algorithms} for the will of the people. This week reveals that these are not neutral measurements—they are engineered \textbf{mechanisms} that define the winner. Like any mechanism, they can be designed, manipulated, and broken. \textbf{The algorithm creates the outcome.}

\section*{Part 1: Scholarly Peer Review (The Original Trust Algorithm)}
Scientific truth isn't determined by popular vote, but by \textbf{Peer Review}—independent experts evaluate ideas before publication.

\textbf{The Hierarchy of Trust.}
Not all publications carry equal weight:
\begin{itemize}[noitemsep, topsep=0pt, leftmargin=*]
    \item \textbf{Pre-prints (e.g., arXiv):} Fast, but unverified. Low trust.
    \item \textbf{Conferences:} Peer-reviewed with faster cycles (common in CS). Medium-High trust.
    \item \textbf{Journals:} Slow, rigorous review. The Gold Standard. High trust.
\end{itemize}

\textbf{Blinding in Peer Review.}
To reduce bias, reviews are often ``blinded'':
\begin{itemize}[noitemsep, topsep=0pt, leftmargin=*]
    \item \textbf{Single-blind:} Reviewers know the authors; authors don't know reviewers.
    \item \textbf{Double-blind:} Neither party knows the other—the most rigorous standard for minimizing bias.
\end{itemize}

\textbf{The Metric: h-index.}
We quantify a researcher's influence using the \textbf{h-index}: A researcher has an index of $h$ if they have published $h$ papers each cited at least $h$ times. This balances \textit{quantity} (productivity) with \textit{quality} (impact), preventing gaming through many uncited papers or a single ``one-hit wonder.''

\textbf{The Hack: Gaming Science.}
Because career advancement depends on metrics, the system is vulnerable to \textbf{Goodhart's Law}: ``When a measure becomes a target, it ceases to be a good measure.'' A study of over 12,000 academics reveals how widespread the problem is:
\begin{itemize}[noitemsep, topsep=0pt, leftmargin=*]
    \item \textbf{Honorary Authorship:} \textbf{35.5\%} admitted adding authors who contributed nothing—often to add a ``big name'' for credibility.
    \item \textbf{Coercive Citations:} \textbf{14.1\%} reported being forced by editors to cite the editor's own journal.
    \item \textbf{Padded Citations:} \textbf{$>$40\%} would add superfluous citations if the journal expected it.
    \item \textbf{Citation Rings:} Groups agree to cite each other regardless of relevance, artificially inflating h-indices.
    \item \textbf{Result:} Bad science appears reputable, misleading the public and policymakers.
\end{itemize}

\section*{Part 2: Search Engine Manipulation (The Modern Gatekeeper)}
If peer review gates scientific truth, search engines gate everyday information. Search rankings dictate reality for the average user:
\begin{itemize}[noitemsep, topsep=0pt, leftmargin=*]
    \item \textbf{The Power of Page 1:} 91.5\% of clicks happen on the first page. If you aren't there, you're invisible.
    \item \textbf{Misplaced Trust:} We assume the top result is ``best'' or ``truest,'' but it's often just the best \textit{optimized}.
\end{itemize}

\textbf{Black Hat SEO (Search Engine Optimization).}
Manipulators use unethical tricks to fool algorithms:
\begin{itemize}[noitemsep, topsep=0pt, leftmargin=*]
    \item \textbf{Cloaking:} Showing one page to the crawler (keyword-optimized) and a different page to humans (spam/ads).
    \item \textbf{Link Farms:} Networks of fake websites created solely to link to a target, exploiting PageRank's ``links as votes.''
    \item \textbf{Invisible Text:} Hiding keywords (white text on white background) so bots index them while users see nothing.
    \item \textbf{Google Bombing:} Coordinated linking of a phrase (e.g., ``miserable failure'') to a target page (e.g., a politician).
    \item \textbf{Bowling:} Harming a competitor by pointing ``bad'' links (from spam sites) at them, triggering a penalty.
\end{itemize}

\textbf{The Search Engine Manipulation Effect (SEME).}
Biased search results don't just change clicks—they change \textit{votes}:
\begin{itemize}[noitemsep, topsep=0pt, leftmargin=*]
    \item \textbf{The Finding:} Biased rankings shifted undecided voters' preferences by \textbf{20\% or more}—up to \textbf{80\%} in some demographics.
    \item \textbf{Real-World Test:} During India's 2014 election, biased results shifted preferences even for candidates voters already knew.
    \item \textbf{Awareness Doesn't Help:} Even knowing results were biased, the effect often persisted. People trust ``top = best.''
    \item \textbf{The Threat:} This is \textbf{invisible gerrymandering}—personalized results mean regulators see neutral pages while targeted voters see bias.
\end{itemize}

\section*{Part 3: The Scorecard (Dahl's Criteria)}
\textbf{How Do Trust Systems Measure Up?}

\textbf{1. Effective Participation (Can you be heard?)}
\textit{Grade:} \textbf{Fail.} Scientists without ``big names'' struggle to publish. Websites without SEO budgets are invisible.

\textbf{2. Voting Equality (One Person, One Vote).}
\textit{Grade:} \textbf{Fail.} Citation rings and link farms create ``fake votes.'' A researcher in a ring counts more than an honest one.

\textbf{3. Enlightened Understanding (Do you know the truth?)}
\textit{Grade:} \textbf{Fail.} When \textbf{Citation Rings}, \textbf{Link Farms}, and \textbf{SEME} manipulate information, citizens make choices based on false data—believing they are informed.

\textbf{4. Control of the Agenda (Who decides what we discuss?)}
\textit{Grade:} \textbf{Fail.} Algorithms and gatekeepers decide what's ``worthy.'' SEO consultants and journal editors—not citizens—set the agenda.

\textbf{5. Inclusiveness (Is everyone invited?)}
\textit{Grade:} \textbf{Mixed.} The internet is open to all, but \textit{visibility} is not. You can publish, but without resources, you won't be found.

\hrulefill

\textbf{Bottom Line.}
Trust systems are not neutral—they are \textbf{designed mechanisms} with rules that can be gamed. Whether it's a citation ring inflating reputation or a link farm boosting rankings, the result is the same: \\textbf{the measure becomes the target}. When search engines can shift votes by 20\%—invisibly and personally—the line between ``ranking information'' and ``manipulating democracy'' vanishes.

\newpage
\section*{Week 5: Election Methods}

\textbf{The Big Picture.}
This week asks: \textit{Does the way we count votes matter as much as who votes?} We often think of Democracy as a set of values. This week teaches us that Democracy is an \textbf{Engineering Problem}. The specific code we use to count votes—the \textbf{Election Method}—determines the winner. Different algorithms applied to the same ballots can produce different winners.

\section*{Part 1: What Makes a Good Election System?}
Before choosing an algorithm, we need to know what we want from it. A good election system should have:

\textbf{Core Properties:}
\begin{itemize}[noitemsep, topsep=0pt, leftmargin=*]
    \item \textbf{Fairness:} Every vote should carry equal weight (Voter Fairness). The system should not favor specific candidates, e.g., based on ballot order (Candidate Fairness).
    \item \textbf{Integrity:} One person, one vote (no double counting). Secret ballots to prevent coercion or bribery.
    \item \textbf{Simplicity:} Easy to understand and execute for the average citizen.
    \item \textbf{Accessibility:} Voting must be accessible to everyone, including those with mobility challenges.
    \item \textbf{Determinism:} The same inputs should always produce the same result (no random winners).
    \item \textbf{Practicality:} The system should be logistically feasible and cost-effective.
\end{itemize}

\section*{Part 2: The Problem with Plurality}
The most common system (used in the US/UK) is \textbf{Plurality} (``First Past the Post'').
\textit{Rule:} Everyone gets one vote. The candidate with the most votes wins.

\textbf{Why it Fails:}
\begin{itemize}[noitemsep, topsep=0pt, leftmargin=*]
    \item \textbf{Vote Splitting:} If two similar candidates (e.g., Two Liberals vs One Conservative) run, they split the liberal vote. The Conservative wins, even if the majority of the country is Liberal.
    \item \textbf{Strategic Voting:} Voters are forced to vote for the ``lesser of two evils'' rather than their true preference.
    \item \textbf{Duverger's Law:} Because of vote splitting, voters are afraid to ``waste'' their vote on a third party. This mathematically forces a rigid \textbf{Two-Party System}.
\end{itemize}

\section*{Part 3: Better Algorithms (Alternative Voting)}
We can fix these bugs by changing the software of democracy.

\textbf{1. Runoff Voting (Two Rounds).}
If no candidate wins a majority ($>$50\%) in the first round, the top two face a second election.
\begin{itemize}[noitemsep, topsep=0pt, leftmargin=*]
    \item \textbf{Benefit:} Ensures the final winner has majority support.
    \item \textbf{Problem:} Expensive (running two elections) and causes \textbf{Voter Fatigue}—turnout often drops significantly in round two.
\end{itemize}

\textbf{2. Instant Runoff Voting (IRV) / Alternative Vote.}
Voters rank candidates. If no one has a majority, the candidate with the \textit{fewest} first-choice votes is eliminated, and their votes transfer to voters' next preferences. Repeat until a winner emerges.
\begin{itemize}[noitemsep, topsep=0pt, leftmargin=*]
    \item \textbf{Benefit:} Simulates a runoff in a single election event. This is the \textit{single-winner} version of STV.
\end{itemize}

\textbf{3. The Condorcet Method (Head-to-Head).}
Imagine a 1-on-1 race between every possible pair of candidates.
\begin{itemize}[noitemsep, topsep=0pt, leftmargin=*]
    \item \textbf{The Winner:} The candidate who beats \textit{everyone else} in these head-to-head match-ups.
    \item \textbf{Benefit:} Selects the ``Consensus Candidate'' that the broadest majority favors (often a Centrist).
    \item \textbf{Problem:} \textbf{Condorcet Cycles}—it's possible to have A beats B, B beats C, C beats A, with no clear winner.
\end{itemize}

\textbf{4. Borda Count (Ranked Scoring).}
Voters rank candidates (1st, 2nd, 3rd). 1st place gets max points, last gets 0.
\begin{itemize}[noitemsep, topsep=0pt, leftmargin=*]
    \item \textbf{Benefit:} Finds the candidate with the highest average approval, even if they aren't the passionate \#1 choice of any group.
\end{itemize}

\textbf{5. Coombs' Method (Eliminate the Most Hated).}
Similar to IRV, but instead of eliminating the candidate with the fewest first-place votes, it eliminates the candidate with the \textbf{most last-place votes}.
\begin{itemize}[noitemsep, topsep=0pt, leftmargin=*]
    \item \textbf{Benefit:} Eliminates polarizing candidates early.
\end{itemize}

\textbf{6. Approval Voting (Simplicity).}
Voters don't rank; they just ``approve'' (check the box) for as many candidates as they like.
\begin{itemize}[noitemsep, topsep=0pt, leftmargin=*]
    \item \textbf{Benefit:} Simple to count. Eliminates vote splitting (you can approve the Green Party AND the Democrat) so you never ``waste'' a vote.
\end{itemize}

\textbf{7. Single Transferable Vote (STV) – The Gold Standard.}
Voters rank candidates. A \textbf{Quota} (minimum votes to win) is calculated. If your \#1 choice is eliminated (has the fewest votes), your vote \textbf{transfers} to your \#2 choice. If your \#1 wins with extra votes (\textbf{Surplus}), those surplus votes also transfer.
\begin{itemize}[noitemsep, topsep=0pt, leftmargin=*]
    \item \textbf{No Wasted Votes:} You can vote for a small party without fear. If they lose, your vote still counts for your backup.
    \item \textbf{Proportional Representation:} If 30\% of the public supports the Green Party, they actually get $\sim$30\% of the seats.
    \item \textbf{Used In:} Ireland (national elections), Australia (Senate), Malta, Cambridge MA (USA).
\end{itemize}

\textbf{The Quota Problem (Hare vs. Droop).}
How many votes are needed to guarantee a seat? Two main formulas exist:
\begin{itemize}[noitemsep, topsep=0pt, leftmargin=*]
    \item \textbf{Hare Quota:} Total Votes / Seats. Simple, but can give unfair results.
    \item \textbf{Droop Quota:} (Total Votes / (Seats + 1)) + 1. The smallest number such that no more candidates than seats can reach it. More fair.
\end{itemize}


\section*{Part 4: The Scorecard (Dahl's Criteria)}
\textbf{Ranking the Voting Algorithms.}
\begin{itemize}[noitemsep, topsep=0pt, leftmargin=*]
    \item \textbf{Effective Participation:} \textbf{Fail (Plurality) / Pass (STV).} Under Plurality, millions of votes for losing candidates are ``wasted.'' Under STV, almost every vote helps elect someone.
    \item \textbf{Voting Equality:} \textbf{Mixed.} Gerrymandering (drawing district lines) rigs the game in Plurality systems. Proportional systems like STV make Gerrymandering impossible.
\end{itemize}

\textbf{Arrow's Impossibility Theorem.}
Kenneth Arrow proved mathematically that \textbf{no voting system is perfect}. Every system has some flaw (e.g., Condorcet can have ``Cycles''). Democracy is about choosing the system with the \textit{least bad} flaws.

\textbf{Gibbard-Satterthwaite Theorem.}
A related proof: \textbf{No ranking-based voting system can guarantee that voters always benefit from voting honestly.} Every system can be ``gamed'' in some scenario. The goal is to make gaming difficult, not impossible.

\textbf{Strategic vs. Honest Voting.}
\begin{itemize}[noitemsep, topsep=0pt, leftmargin=*]
    \item \textbf{Honest Voting:} You vote for your true preference.
    \item \textbf{Strategic Voting:} You vote for a less-preferred candidate because you think your true favorite can't win.
\end{itemize}
\textit{Key Insight:} Plurality \textit{forces} strategic voting (``don't waste your vote!''). Better systems like STV encourage honest voting because your backup choices still count.

\hrulefill

\textbf{Bottom Line.}
To fix democracy, we don't just need better candidates; we need better \textbf{Algorithms}. Replacing Plurality with \textbf{STV} or \textbf{Ranked Choice} would do more to fix polarization and representation than any single election result. But remember Arrow's law: perfection is impossible. The best we can do is choose the system with the least harmful trade-offs.


\newpage
\section*{Week 6: Blockchains, Smart Contracts, and DAOs}

\textbf{The Big Picture.}
This week asks: \textit{Can code replace institutions?} We explore the technical promise to "democratize" society: \textbf{Blockchain}. We examine Bitcoin, Ethereum, and Decentralized Autonomous Organizations (DAOs). The key lesson: technology that \textit{promises} decentralization often \textit{delivers} new forms of centralization (miners, developers, and wealthy token-holders).

\section*{Part 1: The Three Waves of Decentralization}
Technology has promised to "democratize everything" multiple times.

\begin{center}
\begin{tabularx}{\textwidth}{@{}l X X p{3cm}@{}}
\toprule
\textbf{Wave} & \textbf{Promise} & \textbf{Outcome} & \textbf{Failure Mode} \\
\midrule
\textbf{Usenet} (1979) & Free speech; global forums. & Killed by spam/noise. & No moderation. \\
\midrule
\textbf{P2P} (2000s) & Cut out middlemen (Napster). & Centralized streaming (Spotify) won. & Legal/Convenience. \\
\midrule
\textbf{Blockchain} (2008) & Cut out banks/governments. & Industrialized mining; new intermediaries. & Centralization. \\
\bottomrule
\end{tabularx}
\end{center}

\section*{Part 2: Bitcoin \& Proof of Work}
\textbf{Innovation:} Satoshi Nakamoto (2008) solved the \textbf{Double Spend} and \textbf{Sybil} problems using a \textbf{Blockchain} (shared history) and \textbf{Proof of Work} (computational cost to vote).

\textbf{Proof of Work (PoW):} Miners solve useless puzzles to secure the network.
\begin{itemize}[noitemsep, topsep=0pt, leftmargin=*]
    \item \textbf{Why it works:} "Skin in the game." Costly to attack.
    \item \textbf{Why it fails:} \textbf{Energy Waste} (burns electricity like a country) and \textbf{Centralization} (requires industrial hardware/capital).
\end{itemize}

\section*{Part 3: Smart Contracts \& DAOs}
\textbf{Smart Contracts:} Programs on Ethereum (e.g., "Multi-sig wallet"). Code runs automatically on everyone's machine. "Code is Law."
\textbf{DAOs:} Organizations run entirely by code, promising transparency and automation.

\begin{center}
\begin{tabularx}{\textwidth}{@{}l X X@{}}
\toprule
\textbf{Aspect} & \textbf{The Efficiency Promise} & \textbf{The Centralized Reality} \\
\midrule
\textbf{Enforcement} & Rules execute exactly as coded; no human bias. & \textbf{Bugs are Legal:} Exploits are valid under "Code is Law." \\
\midrule
\textbf{Governance} & Voting via tokens; transparent and public. & \textbf{Plutocracy:} Wealthy holders rule. Developers hold "upgrade keys." \\
\midrule
\textbf{Trust} & Trust the code, not the people. & Code is complex; users trust developers blindly. \\
\bottomrule
\end{tabularx}
\end{center}

\section*{Part 4: Case Study – "The DAO" Hack (2016)}
A \$250M fund was drained due to a code bug.
\begin{itemize}[noitemsep, topsep=0pt, leftmargin=*]
    \item \textbf{The Dilemma:} "Code is Law" purists said let the hacker keep it. Pragmatists said "save the money."
    \item \textbf{The Outcome:} Ethereum developers forced a \textbf{Hard Fork} (rewrote history) to bail out investors.
    \item \textbf{The Split:} Purists stayed on the old chain (\textbf{Ethereum Classic}), proving that when stakes are high, human governance overrides code.
\end{itemize}

\section*{Part 5: The Scorecard (Dahl's Criteria)}
How does "Governance by Blockchain" measure up?

\begin{center}
\begin{tabularx}{\textwidth}{@{}p{3cm} X@{}}
\toprule
\textbf{Criterion} & \textbf{Blockchain \& DAOs Grade} \\
\midrule
\textbf{Effective Participation} & \textbf{Fail.} Mining is industrialized. "Permissionless" entry is a myth; you need capital. \\
\midrule
\textbf{Voting Equality} & \textbf{Fail.} Token-weighted voting means \textbf{1 Dollar = 1 Vote}. It is a plutocracy, not a democracy. \\
\midrule
\textbf{Enlightened Understanding} & \textbf{Mixed.} Code is open source, but only experts can audit it. Most users trust blindly. \\
\midrule
\textbf{Control of the Agenda} & \textbf{Fail.} Developers and large whales propose and decide upgrades. \\
\midrule
\textbf{Inclusiveness} & \textbf{Fail.} High technical/financial barriers exclude the majority. \\
\bottomrule
\end{tabularx}
\end{center}

\hrulefill

\textbf{Bottom Line.} Blockchain swaps trusted institutions (Banks) for code. But code is written by humans, and "decentralized" systems inevitably re-centralize around capital (mining pools, large token holders). \textbf{Decentralization is a process, not a product.}

\newpage
\section*{Week 7 \& 8: Beyond Traditional Voting}

\textbf{The Big Picture.} These weeks explore \textbf{innovative alternatives to traditional voting}. We ask: \textit{Can we design better systems for collective decision-making?} We examine five mechanisms: \textbf{Liquid Democracy} (a hybrid of direct and representative democracy), \textbf{Crowdfunding} (community-powered financing), \textbf{Prediction Markets} (betting on truth), \textbf{Quadratic Voting} (expressing how much you care), and \textbf{Deliberative Polling} (combining equality with deliberation). Each uses different incentives—delegation, money, voice credits, or structured discussion—to reveal what people truly want.

\section*{Part 1: The Democracy Dilemma}
Traditional democracy forces a choice between two imperfect systems:
\begin{itemize}[noitemsep, topsep=0pt, leftmargin=*]
    \item \textbf{Representative Democracy:} Saves time; politicians specialize in complex issues. But representatives may not represent you perfectly, and you're stuck with ``party bundles'' of positions you may not all agree with.
    \item \textbf{Direct Democracy:} Maximum control—every voice counts on every issue. But it's exhausting; ``hot-button'' issues dominate while boring-but-important topics get ignored. The Swiss model shows it \textit{can} work, but requires a strong civic culture.
\end{itemize}
The mechanisms below try to escape this trade-off by offering new ways to participate, delegate, or express preferences.

\section*{Part 2: Liquid Democracy (The Hybrid Solution)}
\textbf{Liquid Democracy} (or Delegative Democracy) offers a ``best of both worlds'' approach.

\textbf{How it Works:}
\begin{itemize}[noitemsep, topsep=0pt, leftmargin=*]
    \item \textbf{Vote Directly:} On topics you care about or understand well.
    \item \textbf{Delegate:} Give your vote to a trusted proxy (e.g., a doctor for healthcare, an economist for taxes) for other topics.
    \item \textbf{Flexibility:} You can override or revoke delegation at any time. Delegation can be topic-specific.
\end{itemize}

\textbf{Design Challenges:}
\begin{itemize}[noitemsep, topsep=0pt, leftmargin=*]
    \item \textbf{Transitivity (Trust Chains):} If you delegate to Alice, and Alice delegates to Bob, your vote goes to Bob. \textit{Risk:} Unexpected power concentration. \textit{Solution:} Limit chain length or use ``ranked-choice delegation'' (fallback if your primary doesn't vote).
    \item \textbf{The Super-Voter Problem:} Everyone delegates to one celebrity/expert, creating an ``accidental dictator.'' \textit{Solution:} Allow \textbf{split delegation}—divide your vote (50\% to A, 50\% to B) to spread power.
    \item \textbf{Privacy vs. Accountability:} Delegates must be visible for accountability, but this enables vote-buying. \textit{Solution:} Delegates' votes are public; ordinary citizens' votes remain private.
    \item \textbf{Cycles:} A$\to$B$\to$C$\to$A creates an infinite loop. \textit{Solution:} ``Viscous Democracy''—lose a fraction of power at each hop; loops decay naturally.
\end{itemize}

\textbf{Real-World Experiments:}
\begin{itemize}[noitemsep, topsep=0pt, leftmargin=*]
    \item \textbf{LiquidFeedback (German Pirate Party, 2009--2013):} Super-voters emerged (Power Laws), but acted as \textit{stabilizers} rather than dictators. Delegation was driven by trust and competence, not ideology.
    \item \textbf{Google Votes (2012--2015):} Internal experiment with 20,000 employees. Mostly low-stakes decisions (food, swag). Only \textbf{3.6\%} delegated. Niche experts attracted delegations on specialized topics.
\end{itemize}

\textbf{Golden Rule:} ``If I give you my vote, I can see what you do with it.'' Transparency enables accountability.

\section*{Part 3: Crowdfunding (Community Finance)}
Crowdfunding lets groups raise money directly from the public to finance ideas, products, or community needs. Platforms like Kickstarter and GoFundMe have made this mainstream.

\textbf{How it Works:}
A creator proposes a project; the community funds it through donations or pre-purchases. If enough people contribute, the project gets built. If not, the money is returned.

\textbf{Democratic Use Cases:}
\begin{itemize}[noitemsep, topsep=0pt, leftmargin=*]
    \item \textbf{Niche Products:} Funding items too specific or risky for traditional investors (e.g., indie films, board games).
    \item \textbf{Community Action:} Solving local problems that government ignores (e.g., fixing potholes, building a community garden).
    \item \textbf{Political/Social Tools:} Empowering sidelined minorities, funding political campaigns, or organizing protest actions outside traditional party structures.
\end{itemize}

\textbf{The Risks:}
\begin{itemize}[noitemsep, topsep=0pt, leftmargin=*]
    \item \textbf{Displacing Government:} If citizens crowdfund to fix roads, the government faces less pressure to do its job.
    \item \textbf{Wealth Inequality:} Rich neighborhoods can crowdfund better school equipment; poor neighborhoods cannot. The mechanism amplifies existing inequality.
    \item \textbf{Flashiness over Merit:} Success often depends on marketing and emotional appeal rather than actual need or quality.
    \item \textbf{Scams and Pressure:} Creators face immense pressure to deliver on promises they may not be able to keep.
\end{itemize}

\section*{Part 4: Prediction Markets (Betting on Truth)}
Prediction markets use \textit{money} to aggregate information. Participants buy and sell contracts that pay out based on whether a future event occurs.

\textbf{How it Works:}
A contract pays \$1 if Event X happens (e.g., ``Candidate A wins the election''). If that contract trades at \$0.53, the market ``believes'' there is a \textbf{53\% chance} the event will occur. Participants with better information can profit by trading, driving prices toward the truth.

\textbf{Why it Matters:}
\begin{itemize}[noitemsep, topsep=0pt, leftmargin=*]
    \item \textbf{Skin in the Game:} Real money forces honesty—you bet on what you actually believe, not what sounds good.
    \item \textbf{Aggregating Dispersed Knowledge:} Information about the future is scattered across millions of people. Markets collect and synthesize it faster than any expert panel.
    \item \textbf{Proven Accuracy:} The Iowa Electronic Markets outperformed Gallup polls—error of 1.5 percentage points vs. 2.1 points in the week before elections, and even better further out.
    \item \textbf{Real-World Use:} Major corporations (Google, Intel, Microsoft, Eli Lilly) use internal prediction markets to forecast sales, product launches, and project timelines.
\end{itemize}

\textbf{The Risks:}
\begin{itemize}[noitemsep, topsep=0pt, leftmargin=*]
    \item \textbf{Manipulation (``Whales''):} Wealthy participants can move prices to create false perceptions, potentially becoming self-fulfilling prophecies.
    \item \textbf{Insider Trading:} People with privileged, non-public information can exploit the market unfairly.
    \item \textbf{Legal Gray Areas:} Is it gambling, investing, or voting? U.S. law heavily restricts prediction markets.
    \item \textbf{The Oracle Problem:} Markets need a trusted source to verify the actual outcome. Disagreements about what ``really happened'' can break the market.
    \item \textbf{Wealth Bias:} Prices reflect the beliefs of those with money to bet, not necessarily the general population.
\end{itemize}

\section*{Part 5: Quadratic Voting (Expressing Intensity)}
Traditional voting treats preferences as binary: you're either ``for'' or ``against.'' But some people care passionately about an issue, while others barely care at all. \textbf{Quadratic Voting (QV)} lets voters express \textit{how much} they care.

\textbf{How it Works:}
Every voter receives a budget of ``voice credits.'' You can buy votes on any issue, but the cost increases quadratically:
\begin{itemize}[noitemsep, topsep=0pt, leftmargin=*]
    \item 1 vote costs 1 credit. \quad 2 votes cost 4 credits. \quad 3 votes cost 9 credits. \quad 10 votes cost 100 credits.
\end{itemize}
This creates a trade-off: spend all your credits on one issue you care deeply about, or spread them across many issues you care about moderately.

\textbf{Why Quadratic?}
Mathematically, quadratic pricing is the \textit{only} cost structure that achieves optimal collective outcomes:
\begin{itemize}[noitemsep, topsep=0pt, leftmargin=*]
    \item \textbf{Linear cost} $\to$ Dictatorship of the most intense voter (one person dominates).
    \item \textbf{Very steep cost} $\to$ Everyone casts exactly one vote (back to 1-person-1-vote).
    \item \textbf{Quadratic cost} $\to$ The optimal ``sweet spot'' between these extremes.
\end{itemize}

\textbf{The Promise:}
\begin{itemize}[noitemsep, topsep=0pt, leftmargin=*]
    \item \textbf{Protects Minorities:} A passionate minority can outweigh an indifferent majority on issues they care deeply about.
    \item \textbf{Reveals True Preferences:} Captures the \textit{intensity} of preferences, leading to better collective decisions.
    \item \textbf{Experimental Validation:} Lab experiments show QV produces outcomes much closer to optimal than traditional voting.
\end{itemize}

\textbf{The Risks:}
\begin{itemize}[noitemsep, topsep=0pt, leftmargin=*]
    \item \textbf{Distraction Politics:} Manipulators could introduce ``hot-button'' issues to trick voters into wasting credits, leaving important decisions to insiders.
    \item \textbf{Complexity:} The math is non-intuitive. Many voters may not understand or trust the system.
    \item \textbf{Fairness to Vulnerable Groups:} Minorities fighting for basic rights are forced to spend all their credits on survival, leaving no influence on other issues.
    \item \textbf{Collusion:} Voters could illegally trade or pool credits to bypass the quadratic scaling.
\end{itemize}

\section*{Part 6: Deliberative Polling (Combining Equality and Deliberation)}
Traditional democracy faces a tension: as political equality expanded (more people voting on more issues), \textit{deliberation} declined. Referenda involve less thoughtful discussion than legislative debate. Polls capture ``top-of-the-head'' reactions, not informed opinions. \textbf{Deliberative Polling} tries to give us both.

\textbf{How it Works:}
\begin{enumerate}[noitemsep, topsep=0pt, leftmargin=*]
    \item \textbf{Random Sample:} Draw a representative sample of citizens.
    \item \textbf{Briefing Materials:} Give them balanced information on the issue, vetted by experts for accuracy and fairness.
    \item \textbf{Deliberative Weekend:} Bring participants together to discuss the issue in small groups with trained moderators who ensure civility, balance, and that all viewpoints are heard.
    \item \textbf{Expert Panels:} Participants can ask questions of policymakers and experts with different perspectives.
    \item \textbf{Measure Opinions:} Survey opinions \textit{before} and \textit{after} deliberation to see what changes.
\end{enumerate}

\textbf{Why it Works:}
Deliberative Polling reveals what people would think \textit{if they had time to learn and discuss}—not top-of-the-head reactions, but considered conclusions.

\textbf{Key Findings from Real Experiments:}
\begin{itemize}[noitemsep, topsep=0pt, leftmargin=*]
    \item \textbf{Opinions Change:} Over half of policy items show significant shifts after deliberation. People update their views when given information.
    \item \textbf{Information-Driven Change:} Those who learn the most change the most. Changes are tied to information gains, not social pressure.
    \item \textbf{No Group Bias:} Changes are unrelated to demographics. Women don't just follow men; the less-educated aren't dominated by the educated.
    \item \textbf{No Systematic Polarization:} About half of small groups become more unified; about half become more diverse. No ``law of group polarization'' here.
    \item \textbf{Ordinary Citizens Can Deliberate:} The ``defeatist'' view—that the public is too ignorant to consult—is wrong. Given opportunity and resources, normal people discuss issues thoughtfully.
\end{itemize}

\textbf{Historical Inspiration:}
Deliberative microcosms chosen by lot made key decisions in \textit{ancient Athens}. Deliberative Polling revives this practice, combining random sampling (equality) with structured discussion (quality).

\section*{Part 7: The Scorecard (Dahl's Criteria)}

\begin{center}
\begin{tabularx}{\textwidth}{@{}p{2.2cm} X X X X X@{}}
\toprule
\textbf{Criterion} & \textbf{Liquid Democracy} & \textbf{Crowdfunding} & \textbf{Prediction Markets} & \textbf{Quadratic Voting} & \textbf{Deliberative Polling} \\
\midrule
\textbf{Participation} & \textbf{Pass.} Flexible (vote or delegate). & \textbf{Mixed.} Anyone can propose; flashy wins. & \textbf{Fail.} Money required. & \textbf{Pass.} Everyone gets credits. & \textbf{Pass.} Random selection. \\
\midrule
\textbf{Voting Equality} & \textbf{Mixed.} Super-voters accumulate power. & \textbf{Fail.} Dollars = votes. & \textbf{Fail.} Wealth buys influence. & \textbf{Pass.} If budgets equal. & \textbf{Pass.} Equal by design. \\
\midrule
\textbf{Understanding} & \textbf{Mixed.} Relies on proxy. & \textbf{Mixed.} Quality varies. & \textbf{Pass.} Aggregates info well. & \textbf{Neutral.} No info boost. & \textbf{Pass.} Deliberation educates. \\
\midrule
\textbf{Agenda Control} & \textbf{Mixed.} Super-voters dominate. & \textbf{Pass.} Crowd decides. & \textbf{Mixed.} Creators set topics. & \textbf{Neutral.} Pre-set issues. & \textbf{Mixed.} Organizers choose. \\
\midrule
\textbf{Inclusiveness} & \textbf{Pass.} Low barriers. & \textbf{Mixed.} Wealth favored. & \textbf{Fail.} Capital required. & \textbf{Pass.} Broadly accessible. & \textbf{Pass.} Random + support. \\
\bottomrule
\end{tabularx}
\end{center}

\hrulefill

\textbf{Bottom Line.} Democracy is an engineering problem. Traditional one-person-one-vote is not the only solution—and may not be the best.

\textbf{Liquid Democracy} lets you choose when to participate and when to delegate, matching your engagement to your expertise—but still creates power concentrations.

\textbf{Crowdfunding} lets communities fund their own priorities—but favors the wealthy and flashy.

\textbf{Prediction Markets} aggregate dispersed information with remarkable accuracy—but are vulnerable to manipulation and exclude those without capital.

\textbf{Quadratic Voting} captures the \textit{intensity} of preferences, protecting passionate minorities—but is complex and can be gamed.

\textbf{Deliberative Polling} shows that ordinary citizens \textit{can} make informed, thoughtful decisions when given balanced information and structured discussion—countering the pessimistic view that ``the public is too ignorant.''

None of these mechanisms is perfect. Each has vulnerabilities. But together, they expand our toolkit for collective decision-making beyond the simple ballot. The deeper lesson: \textbf{democracy is not one system—it is an ongoing experiment}. We can design new mechanisms, test them, and learn what works. The goal is not to find the perfect system (Arrow proved that's impossible), but to find \textit{better} systems for different contexts. Delegation, money, incentives, and deliberation are all valid tools—the question is how to use them wisely.

\newpage
\section*{Week 9: Randomness in Democracy}

\textbf{The Big Picture.}
This week asks: \textit{Should we select leaders by lottery instead of election?} We explore an ancient but powerful tool for democracy: \textbf{Randomness}. From Athenian lotteries to Venetian Doge elections to modern jury selection, random selection has been used to prevent corruption, ensure fairness, and give ordinary citizens a voice. But randomness only works if identities are real—which brings us to the \textbf{Sybil Attack} (fake identities) and \textbf{Social Bots} (automated fakes).

\section*{Part 1: Why Randomness?}
At first, picking leaders by lottery seems absurd. But randomness solves real problems:

\textbf{The Core Benefits:}
\begin{itemize}[noitemsep, topsep=0pt, leftmargin=*]
    \item \textbf{Anti-Corruption:} If you don't know who will be selected, you can't bribe them in advance.
    \item \textbf{Fairness:} Everyone has an equal chance. No ``old boys' club'' or insider networks.
    \item \textbf{Fresh Perspectives:} Newcomers aren't captured by special interests or ``the way things are done.''
    \item \textbf{Representativeness:} A random sample looks like the population—diverse in age, gender, background.
\end{itemize}

\textbf{The Core Trade-Off:}
\begin{center}
\begin{tabularx}{\textwidth}{@{}X X@{}}
\toprule
\textbf{Randomness (Anti-Corruption)} & \textbf{Expertise} \\
\midrule
Newbies with no vested interests & Experienced insiders who may be captured or corrupt \\
Hard to bribe or manipulate & Deep knowledge and efficiency \\
Diversity and fresh perspectives & Risk of entrenchment and bias \\
\bottomrule
\end{tabularx}
\end{center}
\textit{Key Insight:} Modern democracies may undervalue randomness. How much expertise do we really need vs. the anti-corruption benefits of random selection?

\section*{Part 2: Historical Uses of Randomness}

\textbf{1. Ancient Athens: The Lottery (Kleroteria).}
Athenian democracy didn't elect most officials—it \textit{randomly selected} them.
\begin{itemize}[noitemsep, topsep=0pt, leftmargin=*]
    \item \textbf{How:} A device called the \textbf{kleroteria} (origin of ``lottery'') randomly chose magistrates (like mayors or police chiefs).
    \item \textbf{Rules:} One-year terms. You could only hold a specific office \textit{once in your lifetime}. Showing \textit{desire} for a position could disqualify you.
    \item \textbf{Why:} Prevents entrenched power. New, unattached people are less likely to abuse authority.
\end{itemize}

\textbf{2. Venice: The Doge Election (1300--1800).}
The Venetian Republic faced rampant factionalism. Their solution? A 10-stage process alternating between \textbf{random selection} and \textbf{deliberation}:
\begin{center}
\begin{tabularx}{\textwidth}{@{}c X@{}}
\toprule
\textbf{Step} & \textbf{Action} \\
\midrule
1--2 & From the Great Council, randomly select 30 $\to$ reduce to 9 \\
3--4 & The 9 deliberate and choose 40 $\to$ randomly reduce to 12 \\
5--7 & The 12 choose 25 $\to$ reduce to 9 $\to$ choose 45 \\
8--10 & Reduce to 11 $\to$ choose 41 $\to$ \textbf{the 41 elect the Doge} \\
\bottomrule
\end{tabularx}
\end{center}
\textit{Why so complex?} Impossible for any faction to predict or control the outcome across so many random rounds. Even if you plant allies at one stage, they may not survive the next cut.

\textbf{3. Trial by Jury (U.S. and Beyond).}
\begin{itemize}[noitemsep, topsep=0pt, leftmargin=*]
    \item \textbf{Purpose:} Ensures fairness and impartiality. Expert judges may carry biases; a randomly selected jury of peers may not.
    \item \textbf{Key Benefit:} Ordinary people may better understand an ordinary person's situation.
\end{itemize}

\textbf{4. Random Vote Counters (Switzerland).}
\begin{itemize}[noitemsep, topsep=0pt, leftmargin=*]
    \item \textbf{Purpose:} Prevents election fraud.
    \item \textbf{Mechanism:} Vote counters are randomly selected \textit{before each election}—impossible to corrupt in advance.
\end{itemize}

\textbf{5. Tie-Breaking by Coin Flip.}
When an election is exactly tied, a coin flip preserves fairness. Neither side gains an advantage; democracy moves forward.

\section*{Part 3: Modern Applications}

\textbf{Risk-Limiting Audits (RLAs).}
A statistical method to verify election integrity without recounting every ballot:
\begin{itemize}[noitemsep, topsep=0pt, leftmargin=*]
    \item Randomly sample a small number of paper ballots.
    \item Count them independently; compare to official results.
    \item Statistical math determines if discrepancies are within chance or indicate fraud.
    \item \textbf{Key Benefit:} Efficient, strong safeguard against counting errors.
\end{itemize}

\textbf{Randomness in STV (Single Transferable Vote).}
When a candidate exceeds the quota, surplus votes must be redistributed. One method: \textit{randomly} select which ballots transfer.
\begin{itemize}[noitemsep, topsep=0pt, leftmargin=*]
    \item \textbf{Pro:} Simple, fair, quick.
    \item \textbf{Con:} Makes the election non-deterministic (different random draws could give different results).
\end{itemize}

\textbf{Deliberative Polling \& Citizens' Assemblies.}
Combines random selection with structured deliberation (covered in Week 8). A random sample of citizens is convened, given balanced information, and asked to discuss and form opinions. The result: what the public \textit{would} think if given time to learn and discuss.

\section*{Part 4: The Sybil Attack (Fake Identities)}
Randomness only works if each ``slot'' is a \textit{real, unique person}. The \textbf{Sybil Attack} breaks this assumption.

\textbf{What is a Sybil Attack?}
A single bad actor creates \textit{many fake identities} to gain disproportionate power in a system that assumes ``one person, one vote.''
\begin{itemize}[noitemsep, topsep=0pt, leftmargin=*]
    \item \textbf{Named after:} The book \textit{Sybil} (1973), about a woman with multiple personalities.
    \item \textbf{Example:} If a lottery selects 10 citizens, and one attacker controls 9 fake identities, that attacker wins the lottery.
\end{itemize}

\textbf{Why It's Hard to Stop (Douceur, 2002):}
\begin{itemize}[noitemsep, topsep=0pt, leftmargin=*]
    \item \textbf{Resources vs. Identities:} If one attacker has $\rho$ times more resources than a normal user, they can create $\rho$ fake identities.
    \item \textbf{Simultaneous Validation Required:} If you check identities one at a time, an attacker can reuse resources and forge unlimited identities. You must validate \textit{everyone at once}.
    \item \textbf{Vouching Doesn't Fully Help:} A ``web of trust'' (like PGP) can be exploited if attackers collude or have enough resources.
\end{itemize}

\textbf{The Bottom Line:}
Without a \textbf{trusted central authority} (like a government ID system), it is nearly impossible to guarantee ``one person, one identity'' in a decentralized system. Randomness-based democracy requires robust identity verification.

\section*{Part 5: Social Bots (Automated Fakes)}
The Sybil attack has gone digital. \textbf{Social bots} are automated accounts that pretend to be human on social media.

\textbf{What Can Bots Do?}
\begin{itemize}[noitemsep, topsep=0pt, leftmargin=*]
    \item \textbf{Manipulate Politics:} Bots were used in the 2010 U.S. elections to smear candidates and spread fake news.
    \item \textbf{Move Markets:} A fake tweet about a White House attack (2013) crashed the stock market instantly.
    \item \textbf{Manufacture Consensus:} Create the illusion that a fringe opinion is widely held (``astroturfing'').
    \item \textbf{Spread Misinformation:} Amplify rumors, conspiracies, and propaganda faster than humans can fact-check.
\end{itemize}

\textbf{Modern Bots Are Sophisticated:}
\begin{itemize}[noitemsep, topsep=0pt, leftmargin=*]
    \item Mimic human circadian rhythms (posting at ``normal'' hours).
    \item Steal real users' photos and bios.
    \item Engage in conversations, comment, and answer questions.
    \item Join trending topics with contextually appropriate content.
\end{itemize}

\textbf{How Do We Detect Bots?}
\begin{center}
\begin{tabularx}{\textwidth}{@{}l X@{}}
\toprule
\textbf{Method} & \textbf{How It Works} \\
\midrule
\textbf{Network-Based} & Bots often cluster together (few links to real users). Detect tight-knit ``sybil'' communities. \\
\textbf{Crowdsourcing} & Humans evaluate profiles (Online Social Turing Test). High accuracy, but expensive and slow. \\
\textbf{Machine Learning} & Train classifiers on features: timing patterns, content, follower ratios, linguistic cues. \\
\textbf{Hybrid} & Combine multiple methods. Detect coordinated, synchronized behavior across accounts. \\
\bottomrule
\end{tabularx}
\end{center}
\textit{The Arms Race:} As detection improves, bots evolve. The race ends only when early detection makes deception too costly.

\section*{Part 6: The Scorecard (Dahl's Criteria)}

\begin{center}
\begin{tabularx}{\textwidth}{@{}p{2.2cm} X X@{}}
\toprule
\textbf{Criterion} & \textbf{Randomness (Lotteries, RLAs, Juries)} & \textbf{Threats (Sybils, Bots)} \\
\midrule
\textbf{Effective Participation} & \textbf{Pass.} Everyone has an equal chance to be selected. & \textbf{Fail.} Fake identities crowd out real voices. \\
\midrule
\textbf{Voting Equality} & \textbf{Pass.} One person, one chance (if identities are real). & \textbf{Fail.} One attacker, many votes. \\
\midrule
\textbf{Enlightened Understanding} & \textbf{Mixed.} Deliberative polls educate; raw lotteries do not. & \textbf{Fail.} Bots spread misinformation. \\
\midrule
\textbf{Agenda Control} & \textbf{Mixed.} Organizers still set topics for citizens' assemblies. & \textbf{Fail.} Bots hijack trending topics. \\
\midrule
\textbf{Inclusiveness} & \textbf{Pass.} Random selection is inherently inclusive. & \textbf{Fail.} Fake accounts exclude real users from visibility. \\
\bottomrule
\end{tabularx}
\end{center}

\hrulefill

\textbf{Bottom Line.}
Randomness is a powerful anti-corruption tool. It ensures fairness and prevents elite capture. But it has a fatal flaw: it relies on \textbf{Unique Identity}. If one attacker can pretend to be a thousand people (Sybil Attack), they can rig any lottery. This leads us to the final challenge: How do we prove we are human in a digital world?


\newpage
\section*{Week 10: Digital Identity \& Proof of Personhood}

\textbf{The Big Picture.}
This week asks: \textit{How do we prove we are human on the internet—without sacrificing privacy?} Democracy relies on a simple promise: \textbf{One Person, One Vote}. In the physical world, our bodies guarantee this—you can only be in one place at once. On the internet, identities are cheap, disposable, and infinite. The goal is \textbf{Proof of Personhood}—verifying that you are a \textit{unique} human, without revealing \textit{who} you are.

\section*{Part 1: The Sybil Problem (Resources vs. Identity)}
A \textbf{Sybil Attack} occurs when one person creates multiple fake identities to subvert a reputation system.

\textbf{1. The Doonesbury Case (IP Addresses aren't People).}
In a famous early internet poll to name a character's college, engineers used ``One Vote per IP Address.''
\begin{itemize}[noitemsep, topsep=0pt, leftmargin=*]
    \item \textbf{The Flaw:} MIT owned a ``Class A'' subnet (millions of IPs). One student wrote a script to cycle through them.
    \item \textbf{The Result:} MIT cast more votes than the rest of the internet combined.
    \item \textbf{Lesson:} \textbf{Resources $\neq$ Identity.} If identity costs money (buying IPs) or skill (coding bots), democracy becomes a plutocracy.
\end{itemize}

\textbf{2. The Platform Incentive.}
Why don't Google or Facebook just verify everyone's ID?
\begin{itemize}[noitemsep, topsep=0pt, leftmargin=*]
    \item \textbf{Friction Kills Growth:} Requiring a passport reduces sign-ups. Platforms prioritize users over truth.
    \item \textbf{Liability:} Storing millions of IDs creates a massive target for hackers (a ``Honey Pot'').
    \item \textbf{Exclusion:} Many marginalized people (refugees) lack official docs.
\end{itemize}

\section*{Part 2: The Death of Proxies}
We try to map users to ``proxies''—things we think map to people. They all fail.

\begin{itemize}[noitemsep, topsep=0pt, leftmargin=*]
    \item \textbf{IP Addresses:} Failed. VPNs, NATs, and IPv6 make IPs meaningless for identity.
    \item \textbf{Email/Phone:} Failed. Easy to automate (sim farms, aliases).
    \item \textbf{CAPTCHAs:} Failed. AI is now better at image recognition than humans. CAPTCHAs now only punish humans (especially the visually impaired) while bots breeze past.
\end{itemize}

\section*{Part 3: Proof of Personhood (The Solutions)}
If we can't use proxies, how do we prove humanity? We need a system where identities are \textbf{Non-Disposable} and \textbf{Unique}, but still \textbf{Private}.

\textbf{1. Pseudonym Parties (Synchronicity).}
\textit{Idea:} You can have 1,000 digital avatars, but only one physical body.
\begin{itemize}[noitemsep, topsep=0pt, leftmargin=*]
    \item \textbf{Mechanism:} Everyone meets at a physical location at the \textbf{exact same time} (Global Synchronicity).
    \item \textbf{Process:} You show up, get a digital token, and leave. No names, no IDs. Just presence.
    \item \textbf{Why it works:} A Sybil attacker cannot send their one body to 1,000 parties simultaneously.
    \item \textbf{Trade-off:} High friction (you have to show up). Hard to organize.
\end{itemize}

\textbf{2. The Web of Trust (Vouching).}
\textit{Idea:} Existing users ``vouch'' for new users (e.g., PGP, keysigning).
\begin{itemize}[noitemsep, topsep=0pt, leftmargin=*]
    \item \textbf{Why it fails:} \textbf{Sybil Regions}. Once a group of attackers gets in, they vouch for each other infinitely (Collusion), creating a massive fake network that looks real.
\end{itemize}

\textbf{3. Biometrics (The Aadhaar Model).}
\textit{Idea:} Use biological uniqueness (Iris, Fingerprint).
\begin{itemize}[noitemsep, topsep=0pt, leftmargin=*]
    \item \textbf{Example:} India's Aadhaar system covers $>$1 billion people.
    \item \textbf{Pros:} Extremely hard to fake.
    \item \textbf{Cons:} \textbf{Zero Privacy.} If the database leaks, you can't reset your fingerprints. It enables perfect state surveillance.
\end{itemize}

\textbf{4. Online Turing Tests (Idena).}
\textit{Idea:} Synchronous \textit{online} puzzles.
\begin{itemize}[noitemsep, topsep=0pt, leftmargin=*]
    \item \textbf{Mechanism:} Everyone logs in at the same time and solves ``Flip Puzzles'' (narrative logic puzzles hard for AI) within a short window.
    \item \textbf{Why it works:} Even if you have scripts, you can't mentally solve 10 narrative puzzles at once.
    \item \textbf{Risk:} AI advancement (LLMs) will eventually break this.
\end{itemize}

\section*{Part 4: The Scorecard (Dahl's Criteria)}
\textbf{How does Proof of Personhood (PoP) impact democracy?}

\textbf{1. Effective Participation (Can you be heard?)}
\textit{Grade:} \textbf{Pass.} Without Sybil protection, real humans are drowned out by bot armies. PoP is essential to restore the human voice in digital spaces.

\textbf{2. Voting Equality (One Person, One Vote).}
\textit{Grade:} \textbf{The Gold Standard.} PoP is the \textit{only} technology that guarantees 1p1v online. Without it, online voting is just ``One Dollar, One Vote'' (buying resources).

\textbf{3. Enlightened Understanding.}
\textit{Grade:} \textbf{Mixed.} Eliminating bots clears out automated disinformation, but it doesn't solve human polarization or filter bubbles.

\textbf{4. Inclusiveness.}
\textit{Grade:} \textbf{The Challenge.} High-friction systems (Pseudonym Parties) exclude the sick, disabled, or busy. Biometric systems exclude those without docs or biological conformity. There is a trade-off between \textbf{Security} and \textbf{Access}.

\hrulefill

\textbf{Bottom Line.}
The internet was built on \textbf{Weak Identities}. This enabled free speech but broke democracy (Sybils). \textbf{Strong Identity} (Passports) fixes Sybils but kills privacy. The future lies in the middle: \textbf{Proof of Personhood}—using synchronicity (Pseudonym Parties) or clever verification to prove \textit{humanity} without revealing \textit{identity}. Until we solve this, digital democracy will always be captured by those with the most servers.

\newpage
\section*{Week 11: Identity and Anonymity}

\textbf{The Big Picture.}
This week asks: \textit{Can we have both accountability and anonymity?} We face a fundamental tension: Democracy needs \textbf{Identity} (to ensure one person, one vote) but Liberty needs \textbf{Anonymity} (to protect free speech and privacy). This week explores the technologies that pull these apart—from \textbf{Self-Sovereign Identity (SSI)} which gives you total control over your credentials, to \textbf{Tor} and \textbf{Mix Nets} which mathematically guarantee that no one knows who you are. The goal is a synthesis: \textbf{Accountable Anonymity}.

\section*{Part 1: The Evolution of Identity (Self-Sovereign Identity)}
We are moving through three ages of digital identity:
\begin{itemize}[noitemsep, topsep=0pt, leftmargin=*]
    \item \textbf{Centralized (Web 1.0):} You have a different username/password for every site. Site owners control your data.
    \item \textbf{Federated (Web 2.0):} You use ``Login with Google.'' Convenient, but Google becomes the surveillance overlord of your life.
    \item \textbf{Self-Sovereign Identity (SSI):} You hold your own credentials in a digital wallet. No central authority can revoke your identity.
\end{itemize}

\textbf{How SSI Works (The Triangle of Trust):}
\begin{enumerate}[noitemsep, topsep=0pt, leftmargin=*]
    \item \textbf{Issuer:} An authority (e.g., University) cryptographically signs a credential (``Alice has a PhD'') and gives it to Alice.
    \item \textbf{Holder (User):} Alice stores this in her private wallet.
    \item \textbf{Verifier:} Alice wants a job. She shows the signed credential to the Employer. The Employer verifies the University's signature on the blockchain.
    \item \textbf{Key Innovation -- Selective Disclosure:} Alice can prove she is ``Over 18'' \textit{without} sharing her birthday or name, using \textbf{Zero-Knowledge Proofs}.
\end{enumerate}

\section*{Part 2: The Architecture of Anonymity}
How do we protect users from surveillance?
\begin{itemize}[noitemsep, topsep=0pt, leftmargin=*]
    \item \textbf{VPNs (Virtual Private Networks):} Trust transfer. You hide from your ISP, but now the VPN provider sees everything. Weak anonymity.
    \item \textbf{Tor (The Onion Router):} \textbf{Three-hop protection}. You encrypt your packet three times.
    \begin{itemize}[noitemsep, topsep=0pt, leftmargin=*]
        \item Node 1 sees \textit{who you are} but not \textit{where you go}.
        \item Node 2 sees nothing.
        \item Node 3 sees \textit{where you go} but not \textit{who you are}.
        \item \textbf{Weakness:} Susceptible to \textbf{Traffic Analysis}. A global adversary watching both ends can match the timing of packets.
    \end{itemize}
    \item \textbf{Mix Nets (Chaumian Mixing):} The gold standard for electronic voting.
    \begin{itemize}[noitemsep, topsep=0pt, leftmargin=*]
        \item Messages are collected in a batch, cryptographically shuffled (like a deck of cards), and sent out in random order.
        \item \textbf{Benefit:} Mathematically breaks the link between sender and receiver.
        \item \textbf{Cost:} \textbf{Latency}. You have to wait for the batch to fill. Great for email/voting, terrible for web browsing.
    \end{itemize}
\end{itemize}

\section*{Part 3: Case Study -- The 4chan Paradox}
What happens when you create a community with \textbf{Absolute Anonymity} and \textbf{Zero Memory}? A study of 4chan's /b/ board reveals the ``survival of the fittest'' meme culture.

\textbf{The Mechanism:}
\begin{itemize}[noitemsep, topsep=0pt, leftmargin=*]
    \item \textbf{Radical Anonymity:} 90\% of posts are ``Anonymous.'' No reputation, no followers, no fame.
    \item \textbf{Extreme Ephemerality:} Threads die in minutes. The average thread stays on the front page for just \textit{5 seconds}.
    \item \textbf{Result:} Content must be shocking, funny, or engaging to survive. If it's boring, it vanishes instantly. This creates a hyper-evolutionary pressure for memes (giving birth to Rickrolling, LOLcats, etc.).
\end{itemize}

\textbf{The Lesson:} Community \textit{can} exist without identity. But without reputation to punish bad behavior, it inevitably drifts toward chaos and toxicity.

\section*{Part 4: The Scorecard (Dahl's Criteria)}
Does Anonymity help Democracy?

\textbf{1. Effective Participation (Can you be heard?)}
\textit{Grade:} \textbf{Pass.} Anonymity is a shield for the vulnerable. Whistleblowers, dissidents in authoritarian regimes, and marginalized groups can speak safely.

\textbf{2. Voting Equality.}
\textit{Grade:} \textbf{Fail.} In an anonymous system, Sybil attacks are trivial. One person can drown out a thousand others. (See Week 10).

\textbf{3. Enlightened Understanding.}
\textit{Grade:} \textbf{Mixed.} Anonymity allows truth-tellers to leak documents (WikiLeaks), but also allows bad actors to spread disinformation without consequence.

\textbf{4. Control of the Agenda.}
\textit{Grade:} \textbf{Fail.} Without identity/reputation, agenda setting is often hijacked by those with the most time (or bots) to spam the system.

\hrulefill

\textbf{Bottom Line.}
We used to think Anonymity was a binary switch (on or off). We now know it's a spectrum.
\textbf{Tor} gives us privacy for browsing.
\textbf{Mix Nets} give us secrecy for voting.
\textbf{SSI} gives us control over our credentials.
The future of democratic technology isn't about destroying anonymity to ensure safety (the surveillance state), nor absolute anonymity (the chaos of 4chan). It is about \textbf{Accountable Anonymity} -- building cryptographic systems where we can prove we are citizens without doxxing ourselves to the world.

\newpage
\section*{Week 12: Polarization and Radicalization}

\textbf{The Big Picture.}
This week asks: \textit{Does the internet bring us together—or tear us apart?} \textbf{Polarization} is the formation of "Us vs. Them" tribes. \textbf{Radicalization} is the process of moving people from normal disagreements to extreme, violent, or hateful views. We explore how algorithms (like YouTube's recommender) and network structures (like social bubbles) don't just reflect these problems—they actively manufacture them.

\section*{Part 1: The Radicalization Pipeline (The YouTube Audit)}
A major 2020 study by Ribeiro et al. audited YouTube to test the "Radicalization Pipeline" hypothesis: Does the algorithm push users into extremism?

\textbf{The Communities:}
They analyzed three layers of political channels:
\begin{itemize}[noitemsep, topsep=0pt, leftmargin=*]
    \item \textbf{Intellectual Dark Web (IDW):} Controversial but mainstream academics discussing taboos (e.g., Jordan Peterson, Joe Rogan).
    \item \textbf{Alt-lite:} Right-wing activists flirting with extreme ideas but denying overt racism (e.g., Proud Boys).
    \item \textbf{Alt-right:} Openly white supremacist and hate groups (e.g., Richard Spencer).
\end{itemize}

\textbf{The Findings:}
\begin{itemize}[noitemsep, topsep=0pt, leftmargin=*]
    \item \textbf{High Migration:} There is a strong data trail of users migrating from mild to extreme. \textbf{40\%} of users commenting on Alt-right videos came directly from the IDW or Alt-lite communities.
    \item \textbf{The Escalator:} The migration is one-way: people drift from moderate to extreme, rarely the other way.
    \item \textbf{The Algorithm's Role:} Recommendations create a navigable path. While the video recommender doesn't usually jump straight to Nazis, the "Recommended Channels" feature bridges the gap, helping users find the next level of extremism easily.
\end{itemize}

\section*{Part 2: Information Gerrymandering}
We know politicians "gerrymander" maps—drawing weird district lines to guarantee they win elections.
A study by Stewart et al. (2019) shows that we do the same thing with \textbf{Information}.

\textbf{How it Works:}
Imagine a group trying to make a decision (Purple Party vs. Yellow Party).
\begin{itemize}[noitemsep, topsep=0pt, leftmargin=*]
    \item \textbf{The Setup:} Even if both sides have equal numbers and equal voting power...
    \item \textbf{The Gerrymander:} If the Purple Party is organized in an "Echo Chamber" (they only talk to each other) while the Yellow Party is scattered, the Purple Party wins. Their influence is concentrated where it matters.
    \item \textbf{The Zealot Effect:} Just a few "Zealot Bots" (who never change their minds) placed in key spots can sway a whole network.
\end{itemize}

\textbf{The Dilemma (Deadlock):}
If both sides try to gerrymander (building their own separate echo chambers), they stop talking entirely. The result isn't a win for anyone—it's \textbf{Deadlock}. Society loses the ability to reach any consensus at all.

\section*{Part 3: Internet Voting (The Swiss Experiment)}
Could technology solve this by making voting easier? Switzerland has been testing \textbf{Internet Voting} (i-voting) for 15 years.

\textbf{The Realities of i-Voting:}
\begin{itemize}[noitemsep, topsep=0pt, leftmargin=*]
    \item \textbf{Slow Adoption:} Even in tech-savvy Switzerland, adoption is crawling. It took 30 years for postal voting to become standard; i-voting is similar.
    \item \textbf{Who uses it?} Domestic usage is low ($\sim$20\%). The real killer app is for \textbf{Expatriates} (citizens living abroad), where usage hits 50-60\% because postal mail is unreliable.
    \item \textbf{The New Digital Divide:} Usage doesn't depend on age or income as much as \textbf{IT Literacy}. Complexity locks out those who aren't tech-savvy.
    \item \textbf{The Trade-off:} Security experts warn that while convenience is high, the risk of undetectable hacking remains the "Sword of Damocles" over the system.
\end{itemize}

\section*{Part 4: The Scorecard (Dahl's Criteria)}
How does technology fare in the age of Polarization?

\textbf{1. Effective Participation (Can you be heard?)}
\textit{Grade:} \textbf{Fail.} The "Radicalization Pipeline" amplifies extreme voices while drowning out moderates. Information Gerrymandering ensures that the best-connected group wins, not the majority.

\textbf{2. Voting Equality (One Person, One Vote).}
\textit{Grade:} \textbf{Mixed.} Internet voting helps some (expats) but excludes others (low-tech users).

\textbf{3. Enlightened Understanding (Do you know the truth?)}
\textit{Grade:} \textbf{Fail.} Algorithms optimize for engagement, which favors outrage and extremism. Users are funneled into echo chambers where they never see opposing facts.

\textbf{4. Control of the Agenda.}
\textit{Grade:} \textbf{Fail.} The agenda is set by "Zealot Bots" and algorithmic black boxes, not by citizens debating freely.

\hrulefill

\textbf{Bottom Line.}
We used to worry that technology would be a "Big Brother" watching us. The reality is subtler: technology is a \textbf{Lens} that distorts reality. Algorithms maximize engagement by feeding us radical content. Social networks gerrymander our attention, trapping us in echo chambers. The result is a polarized society where we don't just disagree on opinions—we disagree on facts. To fix democracy, we don't just need better voters; we need to redesign the \textbf{architecture of influence} itself.

\end{document}
